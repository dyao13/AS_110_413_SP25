\documentclass[titlepage]{article}

\usepackage{preamble}

\begin{document}

\maketitle

\tableofcontents

\newpage \newsection{Introduction.}

\subsection{Section.} Introduction.

\newpage \newsection{Topological Spaces.}

\subsection{Section.} Topological Spaces.

\subsection{Definition.} A topological space is a tuple $(X, \mathcal{T})$ where $X$ is a set and $\mathcal{T}$ is a collection of subsets of $X$ such that 

(a) $\emptyset \in \mathcal{T}$ and $X \in \mathcal{T}$.

(b) $\mathcal{T}$ is closed under arbitrary union.

(c) $\mathcal{T}$ is closed under finite intersection.

\subsection{Definition.} $T$ is finer than $T'$ if $T \supseteq T'$. $T$ is coarser than $T'$ if $T \subseteq T'$. $T$ and $T'$ are comparable if $T \subseteq T'$ or $T' \subseteq T$.

\subsection{Definition.} A basis $\mathcal{B}$ for a topology on $X$ is a colleciton of subsets of $X$ such that 

(a) for each $x \in X$, there exists a $B \in \mathcal{B}$ such that $x \in B$.

(b) if $x \in B_{1} \cap B_{2}$, then there exists a $B_{3} \subseteq B_{1} \cap B_{2}$ such that $x \in B_{3}$.

\subsection{Definition.} Let $\mathcal{B}$ be a basis. The topology $\mathcal{T}$ generated by $|mathcal{B}$ is such that $U$ is open in $X$ if and only if for each $x \in U$, there exists a $B \in \mathcal{B}$ such that $x \in B \subseteq U$.

\subsection{Lemma.} Let $X$ be a set. Let $\mathcal{B}$ be a basis for the topology $\mathcal{T}$ on $X$, Then $\mathcal{T}$ is the collection of unions of elements of $\mathcal{B}$.

\subsection{Proof.} TOOD!

\subsection{Lemma.} Let $X$ be a topological space. Suppose that $\mathcal{C}$ is a collection of open sets of $X$ such that for each open set $U$ of $X$ and each $x \in U$, there is an element $C \in \mathcal{C}$ such that $x \in C \subseteq U$. Then $\mathcal{C}$ is a basis for the topology of $X$.

\subsection{Proof.} TOOD!

\subsection{Lemma.} Let $\mathcal{B}$ and $\mathcal{B}'$ be bases for $\mathcal{T}$ and $\mathcal{T}'$, respectively, on $X$. Then the following are equivalent.

(a) $\mathcal{T}'$ is finer than $\mathcal{T}$.

(b) For each $x \in X$ and each basis element $B \in \mathcal{B}$, there exists a $B' \in \mathcal{B}'$ such that $x \in B' \subseteq B$.

\subsection{Example.} $R_{l}$ and $R_{K}$ are finer than $\R$, but $\R_{l}$ and $\R_{K}$ are incomparable.

\subsection{Definition.} A subbasis $\mathcal{S}$ for a topology on $X$ is a collection of subsets of $X$ whose union equals $X$. The topology generated by the subbasis $\mathcal{S}$ is defined to be the collection $\mathcal{T}$ of all unions of finite intersections of elements of $\mathcal{S}$.

\subsection{Definition.} Let $X$ be a simply ordered set with more than one element. Let $\mathcal{B}$ consist of 

(a) all open intervals $(a, b)$,

(b) all rays $[a_{0}, b)$ where $a_{0}$ is the smallest element of $X$, and

(c) all rays $(a, b_{0}]$ where $b_{0}$ is the largest element of $X$.

The topology generated by $\mathcal{B}$ is called the order topology on $X$.

\subsection{Remark.} The collection of open rays in $X$ is a subbasis for the order topology on $X$.

\subsection{Definition.} Let $X$ and $Y$ be topological spaces. The product topology on $X \times Y$ is the topology generated by the basis of the sets $U \times V$ where $U$ and $V$ are open in $X$ and $Y$, respectively.

\subsection{Theorem.} If $\mathcal{B}$ and $\mathcal{C}$ are bases for $X$ and $Y$, respectively, then the collection of sets $B \times C$ where $B \in \mathcal{B}$ and $C \in \mathcal{C}$ is a basis for the product topology on $X \times Y$.

\subsection{Proof.} TODO!

\subsection{Definition.} Let $\pi_{1}: X \times Y \to X$ be the projection of $X \times Y$ onto $X$ and similarly for $\pi_{2}: X \times Y \to Y$.

\subsection{Theorem.} The collection 
$$\mathcal{S} = \{\pi_{1}^{-1}(U) \mid U \in \mathcal{T}_{1}\} \cup \{\pi_{2}^{-1}(V) \mid V \in \mathcal{T}_{2}\}$$
is a subbasis for the product topology on $X \times Y$, where $\mathcal{T}_{1}$ and $\mathcal{T}_{2}$ are the topologies on $X$ and $Y$, respectively.

\subsection{Proof.} TODO!

\subsection{Definition.} Let $X$ be a topological space with topology $\mathcal{T}$. If $Y$ is a subset of $X$, the collection 
$$T_{Y} = \{U \cap Y \mid U \in \mathcal{T}\}$$
is a topology on $Y$ called the subspace topology on $Y$.

\subsection{Lemma.} If $\mathcal{B}$ is a basis for $X$, then the collection 
$$\mathcal{B}_{Y} = \{B \cap Y \mid B \in \mathcal{B}\}$$
is a basis for the subspace topology on $Y$.

\subsection{Proof.} TODO!

\subsection{Lemma.} Let $Y$ be a subspace of $X$. If $U$ is open in $Y$ and $Y$ is open in $X$, then $U$ is open in $X$.

\subsection{Proof.} TODO!

\subsection{Theorem.} If $A$ is a subspace of $X$ and $B$ is a subspace of $Y$, then the product topology on $A \times B$ is the same as the subspace topology of $A \times B$ as a subspace of $X \times Y$.

\subsection{Proof.} TODO!

\subsection{Theorem.} Let $x$ be an ordered set in the order topology. Let $Y$ be a subset of $X$ that is convex in $X$. Then the order topology on $Y$ is the same as the subspace topology on $Y$ as a subspace of $X$.

\subsection{Proof.} TODO!

\subsection{Theorem.} Let $X$ be a topological space. Then the following properties hold.

(a) $\emptyset$ and $X$ are closed.

(b) Arbitrary intersections of closed sets are closed.

(c) Finite unions of closed sets are closed.

\subsection{Proof.} TODO!

\subsection{Theorem.} Let $Y$ be a subspace of $X$. Then a set is closed in $Y$ if and only if it equals the intersection of a closed set of $X$ with $Y$.

\subsection{Proof.} TODO!

\subsection{Theorem.} Let $Y$ be a subspace of $X$. If $A$ is closed in $Y$ and $Y$ is closed in $X$, then $A$ is closed in $X$.

\subsection{Proof.} TODO!

\subsection{Definition.} The closure $\overline{A}$ of a set $A$ is the intersection of all closed sets that contain $A$.

\subsection{Theorem.} Let $Y$ be a subspace of $X$. Let $A$ be a subset of $Y$. Let $\overline{A}$ denote the closure of $A$ in $X$. Then the closure of $A$ in $Y$ equals $\overline{A} \cap Y$.

\subsection{Proof.} TODO!

\subsection{Theorem.} Let $A$ be a subset of $X$.

(a) $x \in \overline{A}$ if and only if every open set $U$ containing $x$ intersects $A$.

(b) $x \in \overline{A}$ if and only if every basis element $B$ containing $x$ intersects $A$.

\subsection{Proof.} TODO!

\subsection{Definition.} Let $X$ be a topological space. A point $x$ is a limit point of a set $A$ if every neighborhood of $x$ contains a point of $A$ other than $x$. That is, $x$ is a limit point of $A$ if $x \in \overline{A \setminus \{x\}}$.

\subsection{Theorem.} Let $A$ be a subset of $X$. Let $A'$ be the set of all limit points of $A$. Then $\overline{A} = A \cup A'$.

\subsection{Proof.} TODO!

\subsection{Corollary.} A subset of a topological space is closed if and only if it contains all its limit points.

\subsection{Proof.} $A$ is closed if and only if $A = \overline{A}$, which holds if and only if $A' \subseteq A$.

\subsection{Definition.} A sequence $(x_{n})$ in $X$ is said to converge to $x$ if for every neighborhood $U$ of $x$, there exists an $N$ such that for all $n \geq N$, $x_{n} \in U$.

\subsection{Definition.} A topological space $X$ is said to be $T_{1}$ if for every $x$ and $y$ distinct in $X$, there exists a neighborhood of $x$ that does not contain $y$.

\subsection{Theorem.} Every finite set in a $T_{1}$ space is closed.

\subsection{Proof.} TODO!

\subsection{Theorem.} Let $X$ be a $T_{1}$ spae. Let $A$ be a subset of $X$. Then $x$ is a limit point of $A$ if and only if every neighbohrood of $x$ contains infinitely many points of $A$.

\subsection{Definition.} A topological space $X$ is said to be a Hausdorff space if for every $x$ and $y$ distinct in $X$, there exist disjoint neighborhoods of $x$ and $y$.

\subsection{Theorem.} If $X$ is a Hausdorff space, then a sequence of points of $X$ converges to at most one point of $X$.

\subsection{Proof.} TODO!

\subsection{Theorem.} Every simply ordered set is Hausdorff in the order topology.

\subsection{Proof.} TODO!

\subsection{Theorem.}

(a) A subspace of a Hausdorff space is Hausdorff.

(b) A finite product of Hausdorff spaces is Hausdorff.

\subsection{Proof.} TODO!

\subsection{Definition.} A function $f: X \to Y$ is said to be continuous if for every open subset $V$ of $Y$, the preimage $f^{-1}(V)$ is open in $X$. It suffices to only consider the case where $V$ is a basis element of $Y$.

\subsection{Theorem.} Let $f: X \to Y$. Then the following are equivalent.

(a) $f$ is continuous.

(b) For every subset $A$ of $X$, $f(\overline{A}) \subseteq \overline{f(A)}$.

(c) For every closed set $B$ of $Y$, $f^{-1}(B)$ is closed in $X$.

(4) $f(x)$ is continuous at each point $x$. That is, for each $x \in X$ and each neighborhood $V$ of $f(x)$, there is a neighbohrood $U$ of $x$ such that $f(U)\subseteq V$.

\subsection{Proof.} TODO!

\subsection{Defintiion.} If $f: X \to Y$ is a bijection and both $f$ and $f^{-1}$ are continuous, then $f$ is a homeomorphism. The spaces $X$ and $Y$ are said to be homeomorphic.

\subsection{Definition.} Let $f: X \to Y$. Let $Z = f(X)$ be a subspace of $Y$. If the restriction $f: X \to Z$ is a homeomorphism, then $f$ is said to be an imbedding of $X$ in $Y$.

\subsection{Theorem.} Let $X, Y, Z$ be topological spaces.

(a) If $f: X \to Y$ is a constant function, then $f$ is continuous.

(b) If $A$ is a subspace of $X$, the inclusion map $j: A \to X$ is continuous.

(c) If $f: X \to Y$ and $g: Y \to Z$ are continuous, then $g \circ f: X \to Z$ is continuous.

(d) If $f: X \to Y$ is continuous and if $A$ is a subspace of $X$, then the restriction $f \mid A: A \to Y$ is continuous.

(e) If $f: X \to Y$ is continuous and $Z$ is a subspace of $Y$ that contains $f(X)$, then the restriction $g: X \to Z$ is continuous. If $Z$ is a space with $Y$ as a subspace, then the extension $h: X \to Z$ is continuous.

(f) The map $f: X \to Y$ is continuous if $X$ can be written as theunioin of open sets $\{U_{\alpha}\}$ such that $f \mid U_{\alpha}$ is continuous for each $\alpha$.

\subsection{Proof.} TODO!

\subsection{Lemma.} (Pasting Lemma) Let $X = A \cup B$ where $A$ and $B$ are either both open or both closed in $X$. Let $f: A \to Y$ and $g: B \to Y$ be continuous. If $f(x) = g(x)$ for all $x \in A \cap B$, then the function $h:X \to Y$ defined as 
$$h(x) = \begin{cases} f(x) & \text{if } x \in A \\ g(x) & \text{if } x \in B \end{cases}$$
is continuous.

\subsection{Proof.} TODO!

\subsection{Theorem.} Let $f: A \to X \times Y$ be defined as 
$$f(a) = (f_{1}(a), f_{2}(a))$$
where $f_{1}: A \to X$ and $f_{2}: A \to Y$. Then $f$ is continuous and only if if and only if $f_{1}$ and $f_{2}$ are continuous.

\subsection{Proof.} TODO!

\subsection{Definition.} Let $\{X_{\alpha}\}$ for $\alpha \in J$ be an indexed famiily of topological spaces. The basis of all sets of the form 
$$\prod_{\alpha \in J} U_{\alpha}$$
where $U_{\alpha}$ is open in $X_{\alpha}$ is called the box topology. It suffices to only consider the case where $U_{\alpha}$ is a basis element of $X_{\alpha}$.

\subsection{Definition.} Let $\{X_{\alpha}\}$ for $\alpha \in J$ be an indexed famiily of topological spaces. Let $S_{\beta}$ denote the collection 
$$S_{\beta} = \{\pi_{\beta}^{-1}(U_{\beta}) \mid U_{\beta} \text{ is open in } X_{\beta}\}.$$
Let $\mathcal{S}$ denote the union of all such $S_{\beta}$ for $\beta \in J$. The topology generated by the subbasis $\mathcal{S}$ is called the product topology. It suffices to only consider the case where $U_{\beta}$ is a basis element of $X_{\beta}$.

\subsection{Theorem.} Let $\{X_{\alpha}\}$ for $\alpha \in J$ be an indexed family of topological spaces. the product topology has basis consisting of all sets of teh form 
$$B = \prod_{\alpha \in J} U_{\alpha}$$
where $U_{\alpha}$ is open in $X_{\alpha}$ and $U_{\alpha} = X_{\alpha}$ for all but finitely many $\alpha$. It suffices to only consider the case where $U_{\alpha}$ is a basis element of $X_{\alpha}$.

\subsection{Theorem.} let $A_{\alpha}$ be a subspace of $X_{\alpha}$ for each $\alpha \in J$. The product of the $\{A_{\alpha}\}$ is a subspace of the product of the $\{X_{\alpha}\}$ if both products are given the box topology or if both products are given the product topology.

\subsection{Proof.} TODO!

\subsection{Theorem.} If each $X_{\alpha}$ is Hausdorff, then the product of the $\{X_{\alpha}\}$ is Hausdorff in the box topology and in the product topology.

\subsection{Proof.} TODO!

\subsection{Theorem.} Let $\{X_{\alpha}\}$ be an indexed family of topological spaces. Let $A_{\alpha} \subseteq X_{\alpha}$ for each $\alpha$. Then 
$$\prod \overline{A_{\alpha}} = \overline{\prod A_{\alpha}}$$
in both the box topology and the product topology.

\subsection{Proof.} TODO!

\subsection{Theorem.} Let 
$$f: A \to \prod_{\alpha \in J} X_{\alpha}$$
be defined as 
$$f(a) = (f_{\alpha}(a))_{\alpha \in J}.$$
In the product topology, $f$ is continuous if and only if each $f_{\alpha}$ is continuous.

\subsection{Proof.} TODO!

\subsection{Example.} TODO!

\subsection{Definition.} A metric on a set $X$ is a function $d: X \times X \to \R$ such that 

(a) (Nonnegativity) $d(x, y) \geq 0$ for all $x, y \in X$ and $d(x, y) = 0$ if and only if $x = y$.

(2) (Symmetry) $d(x, y) = d(y, x)$ for all $x, y \in X$.

(3) (Triangle Inequality) $d(x, z) \leq d(x, y) + d(y, z)$ for all $x, y, z \in X$.

\subsection{Definition.} Denote the open $\epsilon$-ball centered at $x$ by $B(x, \epsilon)$.

\subsection{Definition.} If $X$ has the metric $d$, then the collection of all $\epsilon$-balls $B(x, \epsilon)$ is a basis for the topology on $X$. The topology generated by the collection of all $\epsilon$-balls is called the metric topology on $X$.

\subsection{Definition.} A topology is said to be metrizable if there exists a metric $d$ on the set $X$ that induces teh topology of $X$. A metric space is a metrizable space $X$ together with a metric $d$ that gives teh topology of $X$.

\subsection{Definition.} Let $(X, d)$ be a metric space. A subset $A$ of $X$ is said to be bounded of there exists a $M \geq 0$ such that $d(x, y) \leq M$ for all $x, y \in A$.

\subsection{Definition.} If $A$ is bounded and nonempty, then the diameter of $A$ is defined as 
$$\text{diam}(A) = \sup_{x, y \in A} d(x, y).$$

\subsection{Theorem.} Let $(X, d)$ be a metric space. Define the standard bounded metric corresponding to $d$ to be $\overline{d}: X \times X \to \R$ by 
$$\overline{d}(x, y) = \min \{d(x, y), 1\}.$$
Then $\overline{d}$ is a metric on $X$ that induces the same topology as $d$.

\subsection{Proof.} TODO!

\subsection{Lemma.} Let $d$ and $d'$ be two metrics on teh set $X$. Let $\mathcal{T}$ and $\mathcal{T}'$ be the topololgies they induce, respectively. Then $\mathcal{T}'$ is finer than $\mathcal{T}$ if and only if for each $x \in X$ and each $\epsilon > 0$, there exists a $\delta > 0$ such that 
$$B_{d'}(x, \delta) \subseteq B_{d}(x, \epsilon).$$

\subsection{Theorem.} The topologies induced by $d$ and $\rho$ on $\R^{n}$ are the same as the product topology on $\R^{n}$.

\subsection{Proof.} TODO!

\subsection{Definition.} Define the uniform metric $\overline{\rho}$ on $\R^{J}$ by the formula 
$$\overline{\rho}(x, y) = \sup_{j \in J} \overline{d}(x_{j}, y_{j}).$$
It induces the uniform topology on $\R^{J}$.

\subsection{Theorem.} The uniform topology on $\R^{J}$ is finer than the product topology and coarser than the box topology. If $J$ is infinite, then there relationsips are strict.

\subsection{Proof.} TODO!

\subsection{Theorem.} Let $\overline{d}$ be the standard bounded metric on $\R$. Define for $\R^{\omega}$ the metric 
$$D(x, y) = \sup \{\frac{\overline{d}(x_{i}, y_{i})}{i}\}.$$
It induces the product topology on $\R^{\omega}$.

\subsection{Proof.} TODO!

\subsection{Theorem.} Let $f: X \to Y$ and let $X$ and $Y$ be metrizable with metrics $d_{X}$ and $d_{Y}$, respectively. Then $f$ is continuous if and only if for every $x \in X$ and every $\epsilon > 0$, there exists a $\delta > 0$ such that $d_{X}(x, y)$ implies $d_{Y}(f(x), f(y)) < \epsilon$.

\subsection{Lemma.} (Sequence Lemma) Let $X$ be a topological space. Let $A \subseteq X$ If there is a sequence of points of $A$ that converges to $x$, then $x \in \overline{A}$. The converse holds if $X$ is metrizable.

\subsection{Proof.} TODO!

\subsection{Theorem.} Let $f: X \to Y$. If $f$ is continuous, then for every convergent sequence $(x_{n}) \to x$ in $X$, the sequence $(f(x_{n}))$ converges to $f(x)$. The converse holds if $X$ iis metrizable.

\subsection{Definition.} Let $f_{n}: X \to Y$ be a sequence of functions from the set $X$ to the metric space $Y$. The sequence $(f_{n})$ converges uniformly to the function $f: X \to Y$ if given $\epsilon > 0$, there exists an $N$ such that 
$$d(f_{n}(x), f(x)) < \epsilon$$
for all $n > N$ and all $x \in X$.

\subsection{Theorem.} (Uniform Limit Theorem) Let $f_{n}: X \to Y$ be a sequence of continuous functions from the topological space $X$ to the metric spcae $Y$. If $(f_{n})$ converges uniformly to $f: X \to Y$, then $f$ is continuous.

\subsection{Proof.} TODO!

\newpage \newsection{Connectedness and Compactness.}

\subsection{Section.} Connectedness and Compactness.

\subsection{Definition.} Let $X$ be a topological space. A separation of $X$ is a paire $(U, V)$ of disjoint nonempty dubsets of $X$ whose union is $X$. A space $X$ is said to be connected if there exists no separation of $X$.

\subsection{Remark.} An equivalent definition is that $X$ is connected if the only clopen sets of $X$ are $\emptyset$ and $X$.

\subsection{Lemma.} If $Y$ is a subspace of $X$, a separation of $Y$ is a pair of disjoint nonempty sets $A$ and $B$ whose union is $Y$,neither of which contains a limit point of the other. The space $Y$ is connected if there exists no separation of $Y$.

\subsection{Proof.} Suppose that $A$ and $B$ form a sepration of $Y$. Then $A$ is both open and closed in $Y$. $A$ is closed, so $A = \overline{A} \cap Y$, that is, $\overline{A} \cap B = \emptyset$. So $B$ contains no limit points of $A$. Similarly, $A$ contains no limit points of $B$.

Conversely, suppose that $A$ and $B$ are disjoint nonemtpy sets whose unioin is $Y$, neither of which contians a limit point of the other. Then $\overline{A} \cap B = \emptyset$ and $\overline{B} \cap A = \emptyset$, so $overline{A} \cap Y = A$ and $\overline{B} \cap Y = B$. Thus $A$ and $B$ are both closed in $Y$ and therefore both open in $Y$.

\subsection{Lemma.} If the sets $C$ and $D$ form a separation of $X$ and if $Y$ is a connected subspace of $X$, then $Y$ lies entirely in $C$ or entirely in $D$.

\subsection{Proof.} The sets $C \cap Y$ and $D \cap Y$ are disjoint nonempty sets whose union is $Y$. If btoh are nonempty, then $Y$ is not connected, a contradiction.

\subsection{Theorem.} The union of a collection of connected subspacesof $X$ that have a point in common is connected.

\subsection{Proof.} Let $\{A_{\alpha}\}$ be a collection of connected subspaces of $X$ that share a point $p$. Let $U$ denote the union of the collection. Suppose, by way of contradiction, that $Y = C \cup D$ is a separation of $Y$. Without loss of generality, suppose that $p \in C$. Since $A_{\alpha}$ is connected, it must lie entirely in $C$. Hence $A_{\alpha} \subseteq C$ for all $\alpha$, so $Y \subseteq C$, a contradiction of the assumption that $D$ is nonempty.

\subsection{Theorem.} Let $A$ be a connected subspace of $X$. If $A \subseteq B \subseteq \overline{A}$, then $B$ is connected.

\subsection{Proof.} Let $A$ be connected and let $A \subseteq B \subseteq \overline{A}$. Suppose that $B = C \cup D$ is a separation of $B$. Assume, without loss of generality, that $A$ lies entirely in $C$ by Lemma 2.5. Then $\overline{A}$ lies entirely in $\overline{C}$, for $\overline{C}$ is disjoint from $D$ by Lemma 2.3. So $B$ does not intersect $D$, a contradiction that $D$ is nonempty.

\subsection{Remark.} That is, if $A$ is a connected subspace, then adjoining some or all of its limit points yields another connected subspace.

\subsection{Theorem.} The image of a connected space under a continuous map is connected.

\subsection{Proof.} Let $f: X \to Y$ be a continuous map. Let $X$ be connected. Let $Z = f(X)$ be the image of $f$. Let $g: X \to Z$ be the restriction of $f$ to the codomain $Z$. It is surjective. Suppose, by way of contradiction, that $Z = A \cup B$ is a separation of $Z$. Then $g^{-1}(A) \cup g^{-1}(B)$ is a separation of $X$, a contradiction.

\subsection{Theorem.} A finite product of connected spaces is connected.

\subsection{Proof.} TODO!

\subsection{Theorem.} A product of connected spaces is connected in the product topology.

\subsection{Proof.} TODO!

\subsection{Example.} $\R^{\omega}$ in the box topology is not connected, for $A$ the set of bounded sequences and $B$ the set of unbounded sequences form a separation of $\R^{\omega}$.

\subsection{Definition.} A simply ordered set $L$ with more than one element is a linear continuum if 

(a) $L$ has the least upper bound property, and 

(b) if $x < y$, there exists $x < z < y$.

\subsection{Theorem.} If $L$ is a linear continuum in the order topology, then $L$ is connected. Moreoever, intervals and rays are connected in $L$.

\subsection{Proof.} TOOD!

\subsection{Corollary.} The real line $\R$ is connected, and so are intervals and rays in $\R$.

\subsection{Proof.} $\R$ is a linear continuum, so Theorem 2.20 applies.

\subsection{Theorem.} (Intermediate Value Theorem) Let $f: X \to Y$ be a continuous map, where $X$ is a connected space and $Y$ is an ordered set in the order topology. If $a$ and $b$ are two points of $X$ and if $r$ is a point of $Y$ between $f(a)$ and $f(b)$, then there exists a $c$ in $X$ such that $f(c) = r$.

\subsection{Proof.} The sets $A = f(x) \cap (-\infty, r)$ and $B = f(X) \cap (r, \infty)$ are disjoint nonemty open sets in $f(X)$. If there exists no $f(c) = r$, then $A$ and $B$ would form a separation of $f(X)$, a contradiction of Theorem 2.12.

\subsection{Example.} The ordered square is a linear continuum.

\subsection{Example.} If $X$ is a well-ordered set, then $X \times [0, 1)$ is a linear continuum.

\subsection{Definition.} A path in $X$ between $x$ and $y$ is a continuous map $f: I \to X$ such that $f(0) = x$ and $f(1) = y$.

\subsection{Definition.} A space $X$ is saidt to be path connected if every pair of points $x$ and $y$ are connected by a path in $X$.

\subsection{Theorem.} A path connected space is connected.

\subsection{Proof.} Suppose, by way of contradiction, that $X = A \cup B$ is a separation of $X$. Let $f: I \to X$ be a path in $X$. $f(I)$ must be connected by Theorem 2.12, so it lies entirely in $A$ or $B$, a contradiction of the path-connectedness of $X$.

\subsection{Example.} The ordered square is connected but not path connected.

\subsection{Example.} Let $S \subseteq \R^{2}$ be the set 
$$S = \{x \times \sin(1/x) \mid 0 < x \leq 1\}.$$
$S$ is connected as the image of a connected space under a continuous map by Theorem 2.12. So it's closure 
$$\overline{S} = S \cup (0 \times I)$$
is also connected by Theorem 2.9. However, $\overline{S}$ is not path connected. Suppose, by way of contradiction, that there exists a path $f: I \to \overline{S}$ such that $f(0) = 0 \times 0$ and $f(1) \in S$. The preimage $f^{-1}(0 \times I)$ is closed, so it has a maximum element $b$. Let $f: [b, 1] \to \overline{S}$ be a path that maps $b$ to $0 \times I$ and maps all other points of $[b, 1]$ to $S$. Replace $[b, 1]$ with $[0, 1]$ for convenience. Let $f(t) = (x(t), y(t))$. There exists a sequence of points $t_{n} \to 0$ such that $y(t_{n}) = (-1)^{n}$, contradicting the continuitiy of $f$ (for the sequence does not converge).

Indeed, given $n$, choose $0 < u < x(1/n)$ such that $\sin(1/u) = (-1)^{n}$. Then use the intermediate value theorem to choose $0 < t_{n} < 1/n$ such that $x(t_{n}) = u$.

\subsection{Definition.} Let $X$ be a topological space. Define the equivalence relation $\sim$ on $X$, where $x \sim y$ if there is a connected subspace of $X$ that cotnains btoh $x$ and $y$. The equivalence classes of $\sim$ are called the (connected) components of $X$.

\subsection{Theorem.} The componenets of $X$ are disjoint subspaces of $X$ whose union is $X$ such that each nonempty connected subspaoce of $X$ intersects only one component of $X$.

\subsection{Proof.} TODO!

\subsection{Definition.} Let $X$ be a topological space. Define teh equivalence relation $\sim_{p}$ on $X$ where $x \sim_{p} y$ if there exists a path in $X$ between $x$ and $y$. The equivalence classes of $\sim_{p}$ are called the path components of $X$.

\subsection{Theorem.} The path components of $X$ are path-connected disjoint subspaces of $X$ whose union is $X$ such that each nonempty path-connected subspace of $X$ intersects only one path component of $X$.

\subsection{Proof.} TODO!

\subsection{Theorem.} Let $X$ be a topological space. The components of $X$ are closed. If $X$ has finitely many components, then the components are open as well.

\subsection{Proof.} The components are closed for they are connected to their closures by Theorem 2.9. If $X$ has finitely many components, then the components are open as well, for they are the complements of a finite union of closed sets.

\subsection{Definition.} A space $X$ is said to be locally connected at $x$ if every enighborhood of $U$ of $x$ contains a connected neighborhood $V$ of $x$. A space $X$ is said to be locally connected if it is locally connected at each of its points.

\subsection{Defintiion.} A space $X$ is said to be locally path connected at $x$ if every neighborhood of $U$ of $x$ contains a path connected neighborhood $V$ of $x$. A space $X$ is said to be locally path connected if it is locally path connected at each of its points.

\subsection{Example.} The topologist's sine curve is connected but not locally connected. For a small enighborhood of $x \in 0 \times I$ contains infinitely many disjoint pieces of the sine curve, so it is not connected.

\subsection{Theorem.} A space $X$ is locally connected if and only if for every open set $U$ of $X$, each component of $Y$ is open in $X$.

\subsection{Proof.}

($\Rightarrow$) Suppose that $X$ is locally connected. Let $U$ be an open set of $X$. Let $C$ be a component of $U$. If $x \in C$, then there exists a connected neighborhood $V$ of $x$ such that $V \subseteq U$. $V$ is connected, so it lies entirely in $C$, so $C$ is open in $X$.

($\Leftarrow$) Let $U$ be an open set in $X$. Suppose that the components of $U$ are open in $X$. Given $x \in X$ and a neighborhood $U$ of $x$, let $C$ be the component of $U$ that contains $x$. $C$ is connected, so it is open in $X$ by hypothesis. So $X$ is locally connected at $x$.

\subsection{Theorem.} A space $X$ is locally path connected if and only if for every open set $U$ of $X$, each path component of $Y$ is open in $X$.

\subsection{Proof.} The proof is similar to that of Theorem 2.45.

\subsection{Theorem.} If $X$ is a topological space, each path comoponent of $X$ lies in a component of $X$. If $X$ is locally path connected, then the components and path comopnents are the same.

\subsection{Theorem.} Let $x \in X$ and let $C$ its component and $P$ be its path component. $P \subseteq C$ because $P$ is connected.

Suppose, by way of contradiction, that $P \neq C$. Let $Q$ denote the union of all path components of $X$ taht are different from $P$ and intersect $C$. Each lies in $C$, so that $C = P \cup Q$. $X$ is locally path connected, so $P$ and $Q$ are disjoint nonempty open sets in $C$, a contradiction of the connectedness of $C$.

\subsection{Definition.} A collection $\mathcal{A}$ of subsets of $X$ is said to be a covering of $X$ if its union is equal to $X$. It is said to be an open covering if each of its elements is open in $X$.

\subsection{Definition.} A space $X$ is said to be compact if every open covering of $X$ admits a finite subcovering of $X$.

\subsection{Definition.} Let $Y$ be a subspace of $X$. A collection $\mathcal{A}$ of subsets of $X$ is said to be a covering of $Y$ if its union contains $Y$.

\subsection{Lemma.} Let $Y$ be a subspace of $X$. Then $Y$ is compact if and only if every covering of $Y$ by open sets of $X$ admits a fintie subcovering of $Y$.

\subsection{Proof.} 

($\Rightarrow$) Let $Y$ be compact and let $\mathcal{A} = \{A_{\alpha}\}$ for $\alpha \in J$ be a covering of $Y$ by sets open in $X$. The collection 
$$\{A_{\alpha} \cap Y \mid \alpha \in J\}$$
is a covering of $Y$ by sets open in $Y$, hence a finite subcollection of $\mathcal{A}$ covers $Y$.

($\Leftarrow$) Let $\mathcal{A}' = \{A_{\alpha}'\}$ be a covering of $Y$ by sets open in $Y$. For each $\alpha$, choose an $A_{\alpha}$ open in $X$ such that 
$$A_{\alpha}' = A_{\alpha} \cap Y.$$
Then $\mathcal{A} = \{A_{\alpha}\}$ is a covering of $Y$ by setes open in $X$. By hypothesis, some finite sobcollection $A_{1}, \ldots, A_{n}$ covers $Y$, so $A_{1}', \ldots, A_{n}'$ covers $Y$.

\subsection{Theorem.} A closed subspace of a compact space is compact.

\subsection{Proof.} Let $Y$ be a closed subspace of a compact space $X$. Let $\mathcal{A}$ be a covering of $Y$ by open sets of $X$. Define 
$$\mathcal{B} = \mathcal{A} \cup \{X - Y\},$$
which is an open covering of $X$. Some finite subcollection of $\mathcal{B}$ covers $X$. If it includes $X \setminus Y$, discard $X \setminus Y$. The remaining collection is a finite subcollection of $\mathcal{A}$ that covers $Y$, so $Y$ is compact by Lemma 2.54.

\subsection{Theorem.} A compact subspace of a Hausdorff space is closed.

\subsection{Proof.} Let $Y$ be a compact subspace of a Hausdorff space $X$. Let $x_{0} \in X \setminus Y$. There exists a neighborhood of $x_{0}$ that is disjoint from $Y$. Indeed, for each $y \in Y$, choose disjoint neighborhoods $U_{y}$ and $V_{y}$ of $x_{0}$ and $y$, respectively. The collection $\{V_{y}\}$ is an open covering of $Y$, so some finite collection $V_{y_{1}}, \ldots, V_{y_{n}}$ covers $Y$. The set 
$$V = \bigcup_{i=1}^{n} V_{y_{i}}$$
is open and contains $Y$ and is disjoint from 
$$U = \bigcap_{i=1}^{n} U_{y_{i}}.$$
So $U$ is a neighborhood of $x_{0}$ that is disjoint from $Y$, so $X \setminus Y$ is open, so $Y$ is closed.

\subsection{Lemma.} If $Y$ is a compact subspace of a Hausdorff space $X$ and if $x_{0} \in X \setminus Y$, then there exist disjoint open sets $U$ and $V$ containing $x_{0}$ and $Y$, respectively.

\subsection{Proof.} The proof is the same as that of Theorem 2.58.

\subsection{Theorem.} The image of a compact space under a continuous map is compact.

\subsection{Proof.} Let $f: X \to Y$ be continuous and let $X$ be compact. Let $\mathcal{A}$ a covering of $f(X)$ by open sets of $Y$. The collection 
$$\{f^{-1}(A) \mid A \in \mathcal{A}\}$$
is an open covering of $X$, so there exists a finnite subcollection $f^{-1}(A_{1}), \ldots, f^{-1}(A_{n})$ that covers $X$. Then the collection $\{A_{1}, \ldots, A_{n}\}$ is a finite subcollection of $\mathcal{A}$ that covers $f(X)$.

\subsection{Theorem.} Let $f: X \to Y$ be a bijective continuous function. If $X$ is compact and $Y$ is Hausdorff, then $f$ is a homeomorphism.

\subsection{Proof.} We show that $f$ is a closed map. Let $A$ be a closed subset of $X$. Then $A$ is compact, so $f(A)$ is compact, by Theorem 2.62. But $Y$ is Hausdorff, so $f(A)$ is closed, by Theorem 2.58.

\subsection{Lemma.} (Tube Lemma) Consider the product space $X \times Y$, where $Y$ is compact. If $N$ is an open set of $X \times Y$ containing the slice $x_{0} \times Y$, then $N$ contains some tube $W \times Y$ about $x_{0} \times Y$, where $W$ is a neighborhood of $x_{0}$ in $X$.

\subsection{Proof.} Let $X$ and $Y$ be spaces, where $Y$ is compact. Suppose that $x_{0} \in X$ and $N$ is an open set of $X \times Y$ containing the slice $x_{0} \times Y$ of $X \times Y$. Cover $x_{0} \times Y$ by basis elements $U \times V$ lying in $N$. The space $x_{0} \times Y$ is compact, being homeomorphic to $Y$, so there is a finite subcover of $x_{0} \times Y$ of basis elements $U_{1} \times V_{1}, \ldots, U_{n} \times V_{n}$. (Assume each basis element intersects $x_{0} \times Y$.) Define 
$$W = \bigcap_{i=1}^{n} U_{i},$$
which is open and contains $x_{0}$.

$U_{1} \times V_{1}, \ldots, U_{n} \times V_{n}$ cover $W \times Y$. Indeed, let $x \times y \in W \times Y$. Consider $x_{0} \times y \in x_{0} \times Y$. Now $x_{0} \times y \in U_{i} \times V_{i}$, for some $i$, so that $y \in V_{i}$. But $x \in U_{j}$ for every $j$ because $x \in W$, so $x \times y \in U_{i} \times V_{i}$, as desired.

$U_{1} \times V_{1}, \ldots, U_{n} \times V_{n}$ lie in $N$, and they cover $W \times Y$, so $W \times Y$ lies in $N$.

\subsection{Theorem.} A product of finitely many compact spaces is compact.

\subsection{Proof.} Let $X$ and $Y$ be commpact spaces. Let $\mathcal{A}$ be an open covering of $X \times Y$. Given $x_{0} \in X$, the slice $x_{0} \times Y$, being homeomorphic to $Y$, so it is covered by a finite subcollection $A_{1}, \ldots, A_{m}$ of $\mathcal{A}$. The union 
$$N = \bigcup_{i=1}^{m} A_{i}$$
is an open set containing $x_{0} \times Y$. By the tube lemma, $N$ contains a tube $W \times Y$ about $x_{0} \times Y$, where $W$ is open in $X$. Then $W \times Y$ is covered by finitely many elements $A_{1}, \ldots, A_{m}$ of $\mathcal{A}$.

For each $x \in X$, choose a neighborhood $W_{x}$ such that $W_{x} \times Y$ can be covered by finitely many elements of $\mathcal{A}$. The collection $\{W_{x}\}$ is an open covering of $X$, so there exists a finite subcollection $W_{x_{1}}, \ldots, W_{x_{n}}$ that covers $X$. The collection $W_{1} \times Y, \ldots, W_{n} \times Y$ is a finite subcollection of $\mathcal{A}$ that covers $X \times Y$, so $X \times Y$ is compact. It follows by induction that a product of finitely many compact spaces is compact.

\subsection{Remark.} The product of infinitely many compact spaces is compact by the Tychonoff theorem, though this result is difficult to prove.

\subsection{Definition.} A collection $\mathcal{C}$ of subsets of $X$ is said to have the fintie intesection property if the intersection of every finite subcollection of $\mathcal{C}$ is nonempty.

\subsection{Theorem.} Let $X$ be a topological space. Then $X$ is compact if and only if for every collection $\mathcal{C}$ of closed sets in $X$ having the finite intersection property, the intersection of all the elements of $\mathcal{C}$ is nonempty.

\subsection{Proof.} TODO!

\subsection{Theorem.} Let $X$ be a simply ordered set having the least upper bound property. In the order topology, each closed interval in $X$ is compact.

\subsection{Proof.} TODO!

\subsection{Corollary.} Every closed interval in $\R$ is compact.

\subsection{Proof.} $\R$ is a simply ordered set having the least upper bound property, so the result follows from Theorem 2.74.

\subsection{Theorem.} (Heine-Borel Theorem) A subspace $A$ of $\R^{n}$ is compact if and only if it is closed and is bounded in the euclidean metric $d$ or the square metric $\rho$.

\subsection{Proof.} It suffices to only consider $\rho$, for 
$$\rho(x, y) \leq d(x, y) \leq \sqrt{n} \rho(x, y)$$
implies that $A$ is bounded under $d$ if and only if it is bounded under $\rho$.

($\Rightarrow$) Suppose that $A$ is compact. Then $A$ is closed by Theorem 2.58. Consider the collection of open sets 
$$\{B_{\rho}(0, m) \mid m \in \Z_{+}\},$$
which union is all of $\R^{n}$. Some finite subcollection covers $A$, so $A \subseteq B_{\rho}(0, M)$ for some finite $M$. Then $\rho(x, y) \leq 2M$ for all $x, y \in A$, so $A$ is bounded.

($\Leftarrow$) Suppose that $A$ is closed and bounded under $\rho$. Suppose that $\rho(x, y) \leq N$ for all $x, y \in A$. Choose a point of $x_{0} \in A$ and let $\rho(x_{0}, 0) = b$. Then $\rho(x, 0) \leq N + b$ for all $x \in A$. If $P = N + b$, then $A \subseteq [-P, P]^{n}$, which is compact by Theorem 2.68. $A$ is also compact by Theorem 2.56.

\subsection{Theorem.} (Extreme Value Theorem) Let $f: X \to Y$ be continuous, where $Y$ is an ordered set in the order topology. If $X$ is compact, then there exist points $c$ and $d$ in $X$ such that $$f(c) \leq f(x) \leq f(d)$$ for all $x \in X$.

\subsection{Proof.} The image $A = f(X)$ is compact by Theorem 2.62. $A$ has a smalllest element $m$ and a largest element $M$, so there exist $f(c) = m$ and $f(c) = M$.

Indeed, if $A$ has no largest element, then teh collection 
$$\{(-\infty, a) \mid a \in A \}$$
is an open covering of $A$. $A$ is compact, so some finite subcollection 
$$\{(-\infty, a_{1}), \ldots, (-\infty, a_{n})\}$$
covers $A$. If $a_{i}$ is the largest of the $a_{1}, \ldots, a_{n}$ then $a_{i}$ belongs to none of these sets, which contradicts the assumption that the sets cover $A$. Hence, $A$ has a largest element. A similar argument shows that $A$ has a smallest element.

\subsection{Definition.} Let $(X, d)$ be a metric space. Let $A$ be a nonempty subset of $X$. For each $x \in X$, define the distance from $x$ to $A$ by 
$$d(x, A) = \inf \{d(x, y) \mid y \in A\}.$$

\subsection{Theorem.} The diameter of a bounded subset $A$ of a metric space $(X, d)$ is 
$$\sup \{d(a_{1}, a_{2}) \mid a_{1}, a_{2} \in A\}.$$

\subsection{Lemma.} (Lebesgue Number Lemma) Let $\mathcal{A}$ be an open covering of a metric space $(X, d)$. If $X$ is compact, then there exists a $\delta > 0$ such that every subset of $X$ having diameter less than $\delta$ is contained in some element of $\mathcal{A}$. $\delta$ is called the Lebesgue number for the covering $\mathcal{A}$.

\subsection{Proof.} Let $\mathcal{A}$ be an open covering of $X$. If $X \in \mathcal{A}$, then any $\delta > 0$ is a Lebesgue number, so assume that $X \not\in \mathcal{A}$.

A finite subcollection $A_{1}, \ldots, A_{n}$ covers $X$. For each $i$, let $C_{i} = X \setminus A_{i}$ and define $f: X \to \R$ by 
$$f(x) = \frac{1}{n} \sum_{i=1}^{n} d(x, C_{i}).$$
$f(x) > 0$ for all $x$. Indeed, given $x \in X$, choose $i$ such taht $x \in A_{i}$. Then choose $\epsilon$ such that $B(x, \epsilon) \subseteq A_{i}$. Then $d(x, C_{i}) \geq \epsilon$, so $f(x) > \epsilon/n$.

Since $f$ is continuous and $X$ is compact, $f$ has a minumum value $\delta > 0$ by the extreme value theorem. $\delta$ is the Lebesgue number. Indeed, let $B$ be a subset of $X$ with diameter less than $\delta$. Choose a point $x_{0} \in B$. Then $B \subseteq B(x_{0}, \delta)$. Now 
$$\delta \leq f(x_{0} \leq d(x_{0}, C_{m}))$$
where $d(x_{0}, C_{m})$ is the largest such $d(x_{0}, C_{i})$. Then $B(x_{0}, \delta) \subseteq X - C_{m} = A_{m}$.

\subsection{Theorem.} (Uniform Continuity Theorem) Let $f: X \to Y$ be a continuous map of the compact metric space $(X, d_{X})$ to the metric space $(Y, d_{Y})$. Then $f$ is uniformly continuous.

\subsection{Proof.} Given $\epsilon > 0$, cover $Y$ by open balls $B(y, \epsilon/2)$ for $y \in Y$. Let $\mathcal{A}$ be the open covering of $X$ by the preimages of the open balls in $Y$. Choose $\delta$ to be the a Lebesgue number for $\mathcal{A}$. Then if $x_{1}$ and $x_{2}$ are two points of $X$ such that $d_{X}(x_{1}, x_{2}) < \delta$, then $f(x_{1})$ and $f(x_{2})$ are in the same open ball, the images $f(x_{1})$ and $f(x_{2})$ lie in some ball $B(y, \epsilon/2)$. so $d_{Y}(f(x_{1}), f(x_{2})) < \epsilon$.

\subsection{Definition.} A point $x \in X$ is said to be an isolated point of $X$ if the one-pint set $\{x\}$ is open in $X$.

\subsection{Theorem.} Let $X$ be a nonempty compact Hasudorff space. If $X$ has no isolated points, then $X$ is uncountable.

\subsection{Proof.} TODO!

\subsection{Corollary.} Every closed interval in $\R$ is uncountable.

\subsection{Proof.} The closed interval $[a, b]$ is a nonempty compact Hausdorff space with no isolated points, so it is uncountable by Theorem 2.89.

\subsection{Definition.} A space $X$ is said to be limit point compact if every finite subset of $X$ has a limit point.

\subsection{Theorem.} Compactness implies limit piotn compactness.

\subsection{Proof.} Let $X$ be a compact space. If $A$ has no limit point, then $A$ is finite.

Indeed, if $A$ has no limit points, then $A$ is closed. For each $a \in A$, choose a neighborhood $U_{a}$ of $a$ such that $U_{a}$ intersects $a$ alone. The space $X$ is covered by the open set $X - A$ and the open sets $\{U_{a}\}$, so a finite subcollection of these sets covers $X$. So $A$ must be finite for each $a \in A$ is contained only in $U_{a}$.

\subsection{Example.} Let $Y$ consist of two points and give $Y$ the indiscrete topology. The space $X = Z_{+} \times Y$ is limit point compact, for every nonempty subset of $X$ has a limit point. It is not compact, for the covering of $X$ by the open sets $U_{n} = \{n\} \times Y$ for $n \in Z_{+}$ has no finite subcovering.

\subsection{Definition.} Let $X$ be a topological space. If $(x_{n})$ is a sequecne of points of $X$ and if $n_{1} < n_{2} < \ldots$ is a sequence of positive integers, then let $(y_{i}) = (x_{n_{i}})$ be a subsequence of $(x_{n})$. A space $X$ is said to be sequentiall ycompact if every sequence of poitns of $X$ has a convergent subsequence.

\subsection{Theorem.} Let $X$ be a metrizable space. Then the following are equivalent.

(a) $X$ is compact.

(b) $X$ is limit point compact.

(c) $X$ is sequentially compact.

\subsection{Proof.} TODO!

\subsection{Definition.} A space $X$ is said to be locally compact at $x$ if there is some compact subspace $C$ of $X$ that contains a neighborhood of $x$. A space $X$ is said to be locally compact if it is locally compact at each of its points.

\subsection{Theorem.} A compact space is locally compact.

\subsection{Proof.} The proof is trivial.

\subsection{Example.} $\R^{n}$ is locally compact. $\R^{\omega}$ is not locally compact.

\subsection{Example.} A simply ordered set $X$ with the least upper bound is locally compact.

\subsection{Theorem.} Let $X$ be a space. Then $X$ is locally compact Hausdorff if and only if there exists a space $Y$ that satisfies the following properties.

(a) $X$ is a subspace of $Y$.

(b) The set $Y \setminus X$ consists of one point.

(3) $Y$ is compact Hausdorff.

If $Y$ and $Y'$ are two such spaces, then there is a homeomorphism of $Y$ with $Y'$ that equals the indentity map on $X$.

\subsection{Proof.} TODO!

\subsection{Definition.} If $Y$ is commpact Hausdorff and $X$ is a proper subspaec of $Y$ whose closure equals $Y$, then $Y$ is said to bea compactification of $X$. If $Y \setminus X$ consists of one point, then $Y$ is said to be a one-point compactification of $X$.

\subsection{Theorem.} Let $X$ be a Hausdorff space. Then $X$ is locally compact if and only if given $x \in X$, and given a neighborhood $U$ of $x$, there is a neighborhood $V$ of $x$ such taht $\overline{V}$ is compact and $\overline{V} \subseteq U$.

\subsection{Proof.}

($\Rightarrow$) $C = \overline{V}$ is a compact set that contains the neighborhood $V$ of $X$, so $X$ is locally compact.

($\Leftarrow$) Suppose that $X$ is locally compact and let $x \in X$ have the neighborhood $U$. Take the one-point compactification $Y$ of $X$ and let $C = Y \setminus U$. $C$ is closed in $Y$, so $C$ is a compact subspace of $Y$. By Lemma 2.60, there exist disjoint open sets $V$ and $W$ that contain $x$ and $C$, respectively. The closure $\overline{V}$ of $V$ in $Y$ is compact. Moreover, $\overline{V}$ is disjoint from $C$, so $\overline{V} \subseteq U$.

\subsection{Corollary.} Let $X$ be a locally compact Hausdorff space. Let $A$ be a subspace of $X$. If $A$ is closed in $X$ or open in $X$, then $A$ is locally compact.

\subsection{Proof.} Suppose that $A$ is closed in $X$. Given $x \in A$, let $C$ be a compact subspace of $X$ that contains a neighborhood $U$ of $x$. Then $C \cap A$ is closed in $C$ and thus compact by Theorem 2.56. It contains a neighborhood $U \cap A$ of $x \in A$. (We need not assume that $X$ is Hasudorff.)

Suppose that $A$ is open in $X$. Given $x \in A$, apply Theorem 2.108 to choose a neighborhood $V$ of $x$ in $X$ such that $\overline{V}$ is compact and $\overline{V} \subseteq A$. Then $C = \overline{V}$ is a compact subspace of $A$ that contains the neighborhood $V$ of $x$ in $A$.

\subsection{Corollary.} A space $X$ is homeomorphic to an open subspace of a compact Hasudroff space if and only if $X$ is locally compact Hasudorff.

\subsection{Proof.} Apply Theorem 2.105 and Corollary 2.112.

\newpage \newsection{Countability and Separation Axioms.}

\subsection{Section.} Countability and Separation Axioms.

\subsection{Definition.} A space $X$ is said to have a countable basis at $x$ if there exists a countable collection $\mathcal{B}$ of neighborhoods of $x$ such that each neighborhood of $x$ contains at least one of the elements of $\mathcal{B}$.

\subsection{Definition.} A space $X$ is said to be first-countable if it has a countable basis at each of its points.

\subsection{Lemma.} (Sequence lemma) Let $X$ be a topological space. Let $A$ be a subset of $X$. If there exists a sequence of points of $A$ converging to $x$, then $x \in \overline{A}$. The converse holds if $X$ if first-countable.

\subsection{Proof.}

($\Rightarrow$) Let $(x_{n})$ be a sequence of points of $A$ converging to $x$. Then, every neighborhood $U$ of $x$ contains all but finitely many points of $(x_{n})$, so $U \cap A$ is nonempty, which means precisely that $x \in \overline{A}$.

($\Leftarrow$) Let $X$ be first-countable. Let $B_{1}, B_{2}, \ldots$ be a countable basis at $x$. Define a new countable basis with $B_{1}' = B_{1}$ and $B_{n+1}' = B_{n+1} \cap B_{n}'$ for $n = 1, 2, \ldots$. For each $n$, choose a point $x_{n} \in B_{n}' \cap A$. Then, $(x_{n})$ is a sequence of points of $A$ converging to $x$.

\subsection{Theorem.} Let $f: X \to Y$. If $f$ is continuous, then for every convergent sequence $x_{n} \to x$ in $X$, the sequence $f(x_{n})$ converges to $f(x)$. The converse holds if $X$ is first-countable.

\subsection{Proof.}

($\Rightarrow$) Let $V$ be a neighborhood of $f(x)$. Then $f^{-1}(V)$ is a neighborhood of $x$ that contains all but finitely many points of $(x_{n})$. Thus, all but finitely many points of $(f(x_{n}))$ are in $V$, so $f(x_{n}) \to f(x)$.

($\Leftarrow$) Let $X$ be first-countable. Let $A \subseteq X$. We want to show $f(\overline{A}) \subseteq \overline{f(A)}$. By the sequence lemma, if $x \in \overline{A}$, then there exists a sequence $(x_{n})$ of points of $A$ converging to $x$ and hence a sequence of points $f(x_{n})$ of $f(A)$ converging to $f(x)$. So $f(x) \in \overline{f(A)}$ by the sequence lemma, which means that $f(\overline{A}) \subseteq \overline{f(A)}$.

\subsection{Definition.} A space $X$ is said to be second-countable if it has a countable basis.

\subsection{Lemma.} If $X$ is second-countable, then any discrete subspace of $X$ is countable.

\subsection{Proof.} Let $Y$ be a discrete subspace of $X$. For each $y \in Y$, choose a basic neighborhood $B_{y}$ that contains only $y$. Then the map $y \mapsto B_{y}$ is injective, so $Y$ is countable.

\subsection{Example.} The uniform topology $\R^{\omega}$ is first-countable but not second-countable. $\R^{\omega}$ is metrizable, so it is first-countable. Let $Y$ be the set of all sequences of $0$s and $1$s. Then $Y$ is a discrete subspace, but $Y$ is uncountable, so $\R^{\omega}$ is not second-countable in the uniform topology.

\subsection{Theorem.} 

(a) A subspace of a first-countable space is first-countable.

(b) A countable product of first-countable spaces is first-countable.

(c) A subspace of a second-countable space is second-countable.

(d) A countable product of second-countable spaces is second-countable.

\subsection{Proof.} The proof is obvious.

\subsection{Definition.} A space $X$ is said to satisfy the Lindlof property if every open cover of $X$ has a countable subcover.

\subsection{Theorem.} A second-countable space is Lindlof.

\subsection{Proof.} Let $\mathcal{A}$ be an open cover of $X$. Let $\{B_{n}\}$ be a countable basis for $X$. For each $n$ for which it is possible, choose $A_{n} \in \mathcal{A}$ such that $B_{n} \subseteq A_{n}$. Then the collection $\{A_{n}\}$ is a countable subcover of $\mathcal{A}$. Indeed, for all $x \in X$, there exists an open $A \in \mathcal{A}$ that contains $x$, which in turn contains a basic neighborhood $B_{n}$ of $x$. So $A_{n}$ is defined and contains $x$.

\subsection{Definition.} A space $X$ is said to be separable if it has a countable dense subset.

\subsection{Theorem.} A second-countable space is separable.

\subsection{Proof.}

Let $\{B_{n}\}$ be a countable basis for $X$. For each $n$, choose an $x_{n} \in B_{n}$. Then the set $D$ of all such $x_{n}$ is countable and dense.

\subsection{Theorem.} Second-countability, Lindlofness, and separabilty are equivalent for a metrizable space.

\subsection{Proof.} See Munkres Section 30 Exercise 5.

\subsection{Definition.} A space $X$ is said to be $T_{1}$ is for every pair of distinct points $x$ and $y$, there exists a neighborhood $U$ of $x$ such that $y \notin U$.

\subsection{Lemma.} A space $X$ is $T_{1}$ if and only if every singleton set $\{x\}$ is closed.

\subsection{Proof.}

($\Rightarrow$) Suppose that $X$ is $T_{1}$. $\{x\}$ is its own closure, for any other point $y$ has a neighborhood $U$ disjoint from $\{x\}$. So $\{x\}$ is closed.

($\Leftarrow$) Suppose that $\{x\}$ is closed. Let $y \in X$ be distinct from $x$. Then there exists a neighborhood $U$ of $y$ that does not contain $x$. So $X$ is $T_{1}$.

\subsection{Definition.} A space $X$ is said to be $T_{2}$, or Hausdorff, if every pair of distinct points $x$ and $y$ have distinct neighborhoods $U$ and $V$, respectively.

\subsection{Definition.} A space $X$ is said to be regular if for every point $x$ and closed set $A$ such that $x \notin A$, there exists disjoint neighborhoods $U$ and $V$, respectively,

\subsection{Definition.} A space $X$ is said to be $T_{3}$ if it is $T_{1}$ and regular.

\subsection{Definition.} A space $X$ is said to be normal if for every pair of disjoint closed sets $A$ and $B$, there exist disjoint neighborhoods $U$ and $V$ of $A$ and $B$, respectively.

\subsection{Definition.} A space $X$ is said to be $T_{4}$ if it is $T_{1}$ and normal.

\subsection{Theorem.} $T_{4}$ implies $T_{3}$ implies $T_{2}$ implies $T_{1}$.

\subsection{Proof.} The proof is obvious.

\subsection{Remark.} For convenience, regular shall refer to $T_{3}$ and normal shall refer to $T_{4}$.

\subsection{Lemma.} Let $X$ be a $T_{1}$ space. $X$ is regular if and only if for every $x \in X$ and every neighborhood $U$ of $x$, there exists a neighborhood $V$ of $x$ such that $\overline{V} \subseteq U$.

\subsection{Proof.}

($\Rightarrow$) Let $x \in X$ and $U$ be a neighborhood of $x$. Let $B = X - U$. Then $A$ is closed, so there exists a neighborhood $V$ of $x$ that is disjoint from a neighborhood $W$ of $B$. Moreover, $\bar{V}$ is disjoint from $B$, and $\bar{V} \subseteq U$, as desired.

($\Leftarrow$) Let $x \in X$ and $B \subseteq X$ be a closed subset disjoint from $x$. Let $U = X - B$. By hypothesis, there exists a neighborhood $V$ of $x$ such that $\bar{V} \subseteq U$. The open sets $V$ and $X - \bar{V}$ are disjoint open sets containing $x$ and $B$, respectively.

\subsection{Lemma.} Let $X$ be a $T_{1}$ space. $X$ is normal if and only if for every closed subset $A$ of $X$ and every neighborhood $U$ of $A$, there exists a neighborhood $V$ of $A$ such that $\overline{V} \subseteq U$.

\subsection{Proof.} The proof is exactly the same as for the previous theorem.

\subsection{Theorem.}

(a) A subspace of a Hausdorff space is Hausdorff.

(b) A product of Hausdorff spaces is Hausdorff.

(c) A subspace of a regular space is regular.

(d) a product of regular spaces is regular.

\subsection{Proof.}

(a) Let $X$ be Hausdorf and let $Y$ be a subspace of $X$. Let $x$ and $y$ be distinct points of $Y$. There exist disjoint neighborhoods $U$ and $V$ of $x$ and $y$, respectively, in $X$. Then $U \cap Y$ and $V \cap Y$ are disjoint neighborhoods of $x$ and $y$, respectively, in $Y$.

(b) Let $\{X_{\alpha}\}$ be a collection of Hausdorff spaces and let $X$ be the product space. Let $x$ and $y$ be distinct points of $X$. Then there exists at least one $\beta$ such that $x_{\beta} \neq y_{\beta}$. Let $U_{\beta}$ and $V_{\beta}$ be disjoint neighborhoods of $x_{\beta}$ and $y_{\beta}$, respectively, in $X_{\beta}$. The sets $\pi_{\beta}^{-1}(U_{\beta})$ and $\pi_{\beta}^{-1}(V_{\beta})$ are disjoint neighborhoods of $x$ and $y$, respectively, in $X$.

(c) Let $X$ be regular and let $Y$ be a subspace of $X$. Let $x \in Y$ and let $B$ be a closed subset of $Y$ that does not contain $x$. Let $\overline{B}$ be the closure of $B$ in $X$. Then $\overline{B} \cap Y = B$, so $x \not\in \overline{B}$. By the regularity of $X$, the exist disjoint neighborhoods $U$ and $V$ of $x$ and $\overline{B}$, respectively, in $X$. Then $U \cap Y$ and $V \cap Y$ are disjoint neighborhoods of $x$ and $B$, respectively, in $Y$.

(d) Let $\{X_{\alpha}\}$ be a collection of regular spaces and let $X$ be the product space. It follows immediately that $X$ is $T_{1}$ by Theorem 2.36a. Let $x \in X$ have the neighborhood $U$ equals the product of $U_{\alpha}$. For each $U_{\alpha}$, let $V_{\alpha}$ be a neighborhood of $x_{\alpha}$ such that $\overline{V_{\alpha}} \subseteq U_{\alpha}$ by Lemma 2.32. If $U_{\alpha} = X_{\alpha}$, then let $V_{\alpha} = X_{\alpha}$. Let $V$ be the product of $V_{\alpha}$. Then $\overline{V} \subseteq U$, so $X$ is regular.

\subsection{Example.} $\R_{K}$ is Hausdorff but not regular.

\subsection{Example.} $\R_{l}$ is normal.

\subsection{Example.} $\R_{l} \times \R_{l}$ is regular but not normal.

\subsection{Theorem.} A regular space with a countable basis is normal.

\subsection{Proof.} Let $X$ be a regular space with a countable basis $\mathcal{B}$. Let $A$ and $B$ be disjoint closed subsets of $X$. For each $x \in A$, there exists a neighborhood $U$ of $x$ and a neighborhood $V$ of $x$ such that $\overline{V} \subseteq U$. For each such $V$, choose a basic neighborhood $U_{x}$ of $x$ contained in $V$. This is a countable covering of $A$ by sets whose closures do not intersect $B$. Index these sets as $\{U_{n}\}$. Similarly, cover $B$ with a countable collection of neighborhoods $\{V_{n}\}$.

Define 
$$U_{n}' = U_{n} - \bigcup_{i=1}^{n} \overline{V_{i}}$$
and 
$$V_{n}' = V_{n} - \bigcup_{i=1}^{n} \overline{U_{i}}.$$
Each such set is open because it is the difference between an open set and a closed set. Furthermore, the collection $\{U_{n}'\}$ is an open cover of $A$ and the collection $\{V_{n}'\}$ is an open cover of $B$. The open sets 
$$U' = \bigcup_{n=1}^{\infty} U_{n}'$$
and 
$$V' = \bigcup_{n=1}^{\infty} V_{n}'$$
are disjoint open sets that contain $A$ and $B$. Indeed, if $x \in U' \cap V'$, then $x \in U_{i}' \cap V_{j}'$ for some $i$ and $j$. Without loss of generality, assume that $i \leq j$. Then $x \in U_{i}$, but $x \not\in \overline{U_{i}}$, hence a contradiction.

\subsection{Theorem.} A metrizable space is normal.

\subsection{Proof.} Let $X$ be a metrizable space with metric $d$. Let $A$ and $B$ be disjoint closed subsets of $X$. For each $a \in A$, choose $\epsilon_{a}$ such that $B(a, \epsilon_{a})$ is disjoint from $B$, and similarly, choose $\epsilon_{b}$ for each $b \in B$. The open sets 
$$U = \bigcup_{a \in A} B(a, \epsilon_{a}/2)$$
and 
$$V = \bigcup_{b \in B} B(b, \epsilon_{b}/2)$$
are disjoint and contain $A$ and $B$, respectively.

\subsection{Theorem.} A compact Hausdorff space is normal.

\subsection{Proof.} Let $X$ be a compact Hausdorff space. Let $A$ and $B$ be disjoint closed subsets of $X$. For each $x \in A$, choose disjoint neighborhoods $U_{x}$ and $V_{x}$ of $x$ and $B$, respectively. The collection $\{U_{x}\}$ is an open cover of $A$, so there exists a finite subcover $U_{1}, \ldots, U_{n}$. It follows that the sets 
$$U = \bigcup_{i=1}^{n} U_{i}$$
and 
$$V = \bigcap_{i=1}^{n} V_{i}$$
are disjoint neighborhoods of $A$ and $B$, respectively.

\subsection{Theorem.} Every well-ordered set is normal in the order topology.

\subsection{Proof.} First observe that any interval $(x, y]$ is open, for if $y$ is not the maximum element, then $(x, y] = (x, y')$ where $y'$ is the immediate successor of $y$. Let $A$ and $B$ be disjoint closed sets in $X$.

Assume that neither $A$ nor $B$ contains the minimum element $a_{0}$ of $X$. For each $a \in A$, choose a neighborhood $(x_{a}, a]$ disjoint from $B$. For each $b \in B$, choose a neighborhood $(y_{b}, b]$ disjoint from $A$. The sets
$$U = \bigcup_{a \in A} (x_{a}, a]$$
and 
$$V = \bigcup_{b \in B} (y_{b}, b]$$
are disjoint open sets containing $A$ and $B$, respectively. Indeed, if $z \in U \cap V$, then $z \in (x_{a}, a] \cap (y_{b}, b]$ for some $a \in A$ and $b \in B$. Without loss of generality, assume that $a < b$. If $a \leq y_{b}$, then the intervals are disjoint. If $a > y_{b}$, then $(y_{b}, b]$ is not disjoint from $A$, a contradiction.

If $A$ contains $a_{0}$, then the set $A - \{a_{0}\}$ is closed and disjoint from $B$, so it admits disjoint open intervals $U$ and $V$ of $A - \{a_{0}\}$ and $B$, respectively. Now the sets $U \cup \{a_{0}\}$ and $V$ are disjoint neighborhoods of $A$ and $B$, respectively.

\subsection{Remark.} Indeed, every ordered set is normal in the order topology.

\subsection{Example.} If $J$ is uncountable, then the product space $\R^{J}$ is not normal. So the product of normal spaces need not be normal. Nor does the subspace of a normal space need be normal (for $\R^{J}$ is homeomorphic to a subspace of $[0, 1]^{J}$).

\subsection{Example.} The product space $S_{\Omega} \times \overline{S_{\Omega}}$ is not normal. So the product of normal spaces need not be normal. Nor does the subspace of a normal space need be normal (for $overline{S_{\Omega}} \times \overline{S_{\Omega}}$ compact Hasudorff and therefore normal).

\subsection{Lemma.} (Urysohn Lemma) Let $X$ be a normal space. Let $A$ and $B$ be disjoint closed subsets of $X$. THen there exists a continuous map $f: X \to [0, 1]$ such that $f(x) = 0$ for all $x \in A$ and $f(x) = 1$ for all $x \in B$.

\subsection{Proof.} TODO!

\subsection{Definition.} If $A$ and $B$ are two subsets of $X$, and if there exists a continuous function $f: X \to [0, 1]$ such that $f(x) = 0$ for all $x \in A$ and $f(x) = 1$ for all $x \in B$, then we say that $A$ and $B$ are separated by a continuous function.

\subsection{Remark.} The Urysohn lemma says that if every pair $A$ and $B$ of disjoint closed subsets can be separated by open sets, then they can be separated by a continuous function. The converse is trivial.

\subsection{Definition.} A space $X$ is said to be completely regular if it is $T_{1}$ and if for every $x_{0} \in X$ and every closed set $A$ that does not contain $x_{0}$, there exists a continuous function $f: X \to [0, 1]$ such that $f(x_{0}) = 0$ and $f(x) = 1$ for all $x \in A$.

\subsection{Remark.} Complete regularity is also known as $T_{3+1/2}$.

\subsection{Theorem.}

(a) A subspace of a completely regular space is completely regular.

(b) A product of completely regular spaces is completely regular.

\subsection{Proof.}

(a) Let $X$ be completely regular and let $Y$ be a subspace of $X$. Let $x_{0} \in Y$ and let $A$ be a closed subset of $Y$ that does not contain $x_{0}$. Then $A = \overline{A} \cap Y$, where $\overline{A}$ is the closure of $A$ in $X$, and moreoever, $x_{0} \not\in \overline{A}$. There exists a continuous function $f: X \to [0, 1]$ such that $f(x_{0}) = 0$ and $f(x) = 1$ for all $x \in \overline{A}$. Then $f \mid Y$ is a continuous function from $Y$ to $[0, 1]$ such that $f(x_{0}) = 0$ and $f(x) = 1$ for all $x \in A$.

(b) Let $X$ be the product of completely regular spaces $\{X_{\alpha}\}$. Let $b \in X$ and let $A$ be a closed set of $X$ that does not contain $b$. Choose a basic neighborhood $U$ of $b$, and denote $U_{1}, \ldots, U_{n}$ the basic neighborhoods of $b_{\alpha}$ in $X_{\alpha}$ that are not all of $X_{\alpha}$. For each $U_{i}$, choose a continuous function $f_{i}: X_{\alpha} \to [0, 1]$ such that $f_{i}(b_{\alpha}) = 0$ and $f_{i}(x) = 1$ for all $x \in A$. The function $f: X \to [0, 1]$ defined by 
$$f(x) = \prod_{i=1}^{n} f_{i}(x_{\alpha})$$
is continuous and satisfies the desired properties.

\subsection{Theorem.} (Urysohn Metrization Theorem) A regular space with a countable basis is metrizable.

\subsection{Proof.} TODO!

\subsection{Theorem.} (Embedding Theorem) Let $X$ be a a $T_{1}$ space. Suppose that $\{f_{\alpha}\}$ is a family of continuous functions indexed for $\alpha \in J$. If $f_{\alpha}: X \to \R$ satisfying the requirement that for each point $x_{0} \in X$ and each neighborhood $U$ of $x_{0}$, there is an index $\alpha$ such that $f_{\alpha}(x_{0}) > 0$ and $f_{\alpha}(x) = 0$ for all $x \not\in U$. Then the function $F: X \to \R^{J}$ defined by 
$$F(x) = \prod_{\alpha \in J} f_{\alpha}(x)$$
is an imbedding of $X$ in $\R^{J}$. If $f_{\alpha}$ maps $X$ into $[0, 1]$, then $F$ is an imbedding of $X$ in $[0, 1]^{J}$.

\subsection{Proof.} TODO!

\subsection{Theorem.} A space $X$ is completely regular if and only if it is homeomorphic to a subspace of $[0, 1]^{J}$ for some $J$.

\subsection{Theorem.} (Tietze Extension Theorem) Let $X$ be a normal space. Let $A$ be a closed subspace of $X$. Any continuous map of $A$ into $[0, 1]$ or $A$ into $\R$ can be extended to a continuous map of $X$ into $[0, 1]$ or $\R$, respectively.

\subsection{Proof.} TODO!

\newpage \newsection{Fundamental Group.}

\subsection{Section.} Fundamental Group.

\subsection{Definition.} If $f$ and $f'$ are continuous maps of $X$ into $Y$, we say that $f$ is homotopic to $f'$ if there exists a continuous map $F: X \times I \to Y$ such that $F(x, 0) = f(x)$ and $F(x, 1) = f'(x)$ for all $x \in X$. We denote $f \simeq f'$ and call $F$ a homotopy between $f$ and $f'$.

\subsection{Definition.} If $f \simeq f'$ and $f'$ is a constant map, then we say that $f$ is nullhomotopic.

\subsection{Definition.} If $f$ and $f'$ are paths in $X$ with initial point $x_{0}$ and final point $x_{1}$, then we say that $f$ is path homotopic to $f'$ if there exists a continuous map $F: I \times I \to X$ such that $F(x, 0) = f(x)$ and $F(x, 1) = f'(x)$ for all $x \in I$ and $F(0, t) = x_{0}$ and $F(1, t) = x_{1}$ for all $t \in I$. We denote $f \simeq_{p} f'$ and call $F$ a path homotopy between $f$ and $f'$.

\subsection{Lemma} The relations $\simeq$ and $\simeq_{p}$ are equivalence relations.

\subsection{Proof.} The proof is obvious.

\subsection{Definition.} Denote $[f]$ the equivalence class of $f$ under the relation $\simeq_{p}$. $[f]$ is called the path homotopy class of $f$.

\subsection{Example.} If $A$ is a convex subspace of $\R^{n}$, then any two paths in $A$ with the same endpoints are homotopic. For the straihgt-line homotopy 
$$F(x, t) = (1 - t)f(x) + tf'(x)$$
is a homotopy between $f$ and $f'$.

\subsection{Definition.} If $f$ is a path in $X$ from $x_{0}$ to $x_{1}$ and $g$ is a path in $X$ from $x_{1}$ to $x_{2}$, then the product $h = f * g$ is defined as 
$$h(s) = \begin{cases}
f(2s) & \text{if } s \in [0, 1/2] \\
g(2s - 1) & \text{if } s \in [1/2, 1].
\end{cases}$$

\subsection{Definition.} The product between path-homotopy classes is defined as $[f] * [g] = [f * g]$. Indeed, let $F$ be a path homotopy between $f$ and $f'$ and let $G$ be a path homotpy between $g$ and $g'$. Define 
$$H(s, t) = \begin{cases}
    F(2s, t) & \text{if } s \in [0, 1/2] \\
    G(2s - 1, t) & \text{if } s \in [1/2, 1].
\end{cases}$$

\subsection{Lemma.} If $k: X \to Y$ is a continuous map and if $F$ is a path homotopy in $X$ between $f$ and $f'$, then $k \circ F$ is a path homotopy in $Y$ between $k \circ f$ and $k \circ f'$.

\subsection{Proof.} The proof is obvious.

\subsection{Lemma.} If $k: X \to Y$ is a continuous map and if $f$ and $g$ are paths in $X$ such that $f(1) = g(0)$, then 
$$k \circ (f * g) = (k \circ f) * (k \circ g).$$

\subsection{Proof.} The proof is obvious.

\subsection{Theorem.} The operation $*$ satisfies the following properties.

(a) (Associativity) $[f] * ([g] * [h]) = ([f] * [g]) * [h]$ whenever the products are defined.

(b) (Identity) Let $e_{x}$ denote the constant path at point $x$. If $f$ is a path from $x_{0}$ to $x_{1}$, then $[f] * [e_{x_{1}}] = [f]$ and $[e_{x_{0}}] * [f] = [f]$.

(c) (Inverse) If $f$ is a path from $x_{0}$ to $x_{1}$, then let its reverse $\overline{f}$ be defined as $\overline{f}(s) = f(1 - s)$ for $s \in I$. Then $[f] * [\overline{f}] = [e_{x_{0}}]$ and $[\overline{f}] * [f] = [e_{x_{1}}]$.

\subsection{Proof.}

(a) Let $[a, b]$ and $[c, d]$ be two intervals in $I$. There exists an unique continuous map $p: [a, b] \to [c, d]$ of the form $p(x) = mx + k$ called the positive linear map. The inverse of a positive linear map is a positive linear map, and the composition of two positive linear maps is a positive linear map.

When the triple product $f * g * h$ is defined, it is the path $k_{ab}$ in $X$ where on $[0, a]$, it is the positive linear map of $[0, a]$ to $[0, 1]$ followed by $f$ and similarly for $[a, b]$ and $[b, 1]$. The path homotopy class of $k_{ab}$ is independent of the choice of $a$ and $b$. Indeed, $[f] * ([g] * [h])$ is the path homotopy class of $k_{ab}$ where $a = 1/2$ and $b = 3/4$ whille $([f] * [g]) * [h]$ is the path homotopy class of $k_{ab}$ where $a = 1/4$ and $b = 1/2$. These are equivalent.

(b) Let $e_{0}$ denote the constant path in $I$ at $0$ and let $i$ denote the identity path in $I$. Then $e_{0} * i$ is a path in $I$ from $0$ to $1$. $I$ is convex, so there is a path homotopy $G$ between $i$ and $e_{0}*i$, so $f \circ G$ is a path homotopy between $f \circ i = f$ and 
$$f \circ (e_{0} * i) = (f \circ e_{0}) * (f \circ i) = e_{x_{0}} * f.$$
The proof for right identity is entirely similar.

(c) Let $i$ denote the identity path in $I$ and $\overline{i}$ denote its reverse. $I$ is convex, so there is a path homotopy $H$ between $e_{0}$ and $i * \overline{i}$. Then $f \circ H$ is a path homotopy between $f \circ e_{0} = e_{x_{0}}$ and 
$$f \circ (i * \overline{i}) = (f \circ i) * (f \circ \overline{i}) = f * \overline{f}.$$
The proof for left inverse is entirely similar.

\subsection{Theorem.} Let $f$ be a path in $X$. Let $a_{0}, \ldots, a_{n}$ such that that $0 < a_{0} < \ldots < a_{n} < 1$. Let $f_{i}$ be the path in $X$ defined as the positive linear map of $I$ to $[a_{i-1}, a_{i}]$ followed by $f$. Then 
$$[f] = [f_{1}] * \ldots * [f_{n}].$$

\subsection{Proof.} The proof is sketched in Proof 3.15a.

\subsection{Definition.} Let $G$ and $G'$ be two groups with the operation $\cdot$. A homomorphism $f: G \to G'$ is such that 
$$f(x \cdot y) = f(x) \cdot f(y)$$
for all $x, y \in G$. $f$ satisfies $f(e) = e'$ and $f(x^{-1}) = f(x)^{-1}$ for all $x \in G$.

\subsection{Definition.} The kernel of $f$ is the set $f^{-1}(e')$. It is a subgroup of $G$.

\subsection{Definition.} The image of $f$ is the set $f(G)$. It is a subgroup of $G'$.

\subsection{Definition.} A homomorphism is called a monomorphism if it is injective (or equivalently of $f^{-1}(e') = e$). It is called an epimorphism if it is surjective. It is called an isomorphism if it is bijective.

\subsection{Definition.} Let $H$ be a subgroup of $G$. Let $xH$ denote the set of products $xh$ for all $h \in H$. It is called the left coset of $H$ in $G$, and the collection of all such $xH$ for $x \in G$ is a partition of $G$. Similarly, let $Hx$ denote the right coset of $H$ in $G$.

\subsection{Definition.} $H$ is said to be a normal subgroup of $G$ if $xhx^{-1} \in H$ for all $x \in G$ and $h \in H$. In this case, $xH = Hx$ for all $x \in G$. The partition $G/H$ is called the quotient group of $G$ by $H$ with the operation $(xH)(yH) = xyH.$

\subsection{Defintion.} The map $f: G \to G/H$ defined by $f(x) = xH$ is epimorphism with kernel $H$. Conversely, if $f: G \to G'$ is an epimorphism and $N$ is a normal subgroup of $G$, then $f$ induces an isomorphism $g: G/N \to G'$ defined by $g(xN) = f(x)$.

\subsection{Definition.} If $H$ is not normal, then $G/H$ denotes the collection of right cosets of $H$ in $G$.

\subsection{Definition.} Let $X$ be a topological space and let $x_{0} \in X$. The fundamental group $\pi_{1}(X, x_{0})$ relative to the base point $x_{0}$ is the group of path homotopy classes of loops in $X$ based at $x_{0}$ with the operation $*$.

\subsection{Example.} Any convex subspace of $\R^{n}$ has a trivial fundamental group.

\subsection{Definition.} Let $\alpha$ be a path in $X$ from $x_{0}$ to $x_{1}$. Define the map $\hat{\alpha}: \pi_{1}(X, x_{0}) \to \pi_{1}(X, x_{1})$
by 
$$\hat{\alpha}([f]) = [\overline{\alpha}] * [f] * [\alpha].$$
If $f$ is a loop at $x_{0}$, then $\overline{\alpha} * f * \alpha$ is a loop at $x_{1}$.

\subsection{Theorem.} The map $\hat{\alpha}$ is an isomorphism of $\pi_{1}(X, x_{0})$ and $\pi_{1}(X, x_{1})$.

\subsection{Proof.} We compute that 
\begin{align*}
    \hat{\alpha}([f]) * \hat{\alpha}([g]) &= [\overline{\alpha}] * [f] * [\alpha] * [\overline{\alpha}] * [g] * [\alpha] \\
                                          &= [\overline{\alpha}] * ([f] * [g]) * [\alpha] \\
                                          &= \hat{\alpha}([f] * [g]),
\end{align*}
so $\hat{\alpha}$ is a homomorphism. We show that $\hat{\alpha}$ has a left inverse and a right inverse. Let $\beta = \overline{\alpha}$. Then 
\begin{align*}
    \hat{\beta}(\hat{\alpha}([f])) &= [\overline{\beta}] * ([\overline{\alpha}] * [f] * [\alpha]) * [\beta] \\
                                   &= [e_{x_{0}}] * [f] * [e_{x_{1}}] \\
                                   &= [f]
\end{align*}
for all $[f] \in \pi_{1}(X, x_{0})$. A similar computation shows that $\hat{\beta}$ is also a right inverse of $\hat{\alpha}$. So $\hat{\alpha}$ is an isomorphism.

\subsection{Corollary.} If $X$ is path connected and $x_{0}, x_{1} \in X$, then $\pi_{1}(X, x_{0})$ is isomorphic to $\pi_{1}(X, x_{1})$.

\subsection{Proof.} The proof is trivial.

\subsection{Remark.} We must still specify the base point $x_{0}$ in the definition of the fundamental group, for the isomorphism $\hat{\alpha}$ depends on the choice of $\alpha$.

\subsection{Definition.} A space $X$ is said to be simply connected if it is path connected and $\pi_{1}(X, x_{0})$ is trivial for one (and hence all) $x_{0} \in X$.

\subsection{Lemma.} If $X$ is simply connected, then any two paths in $X$ with the same endpoints are path homotopic.

\subsection{Proof.} Let $\alpha$ and $\beta$ be two paths in $X$ from $x_{0}$ to $x_{1}$. We compute that 
\begin{align*}
    [\alpha] &= [\alpha] * [\overline{\beta}] * [\beta] \\
             &= [\alpha * \overline{\beta}] * [\beta] \\
             &= [e_{x_{0}}] * [\beta] \\
             &= [\beta].
\end{align*}

\subsection{Definition.} Let $h: (X, x_{0}) \to (Y, y_{0})$ be a continuous map. The homomorphism $h_{*}: \pi_{1}(X, x_{0}) \to \pi_{1}(Y, y_{0})$ induced by $h$, relative to base point $x_{0}$, is defined as 
$$h_{*}([f]) = [h \circ f].$$
$h_{*}$ is a homomorphism because 
$$h \circ (f * g) = (h \circ f) * (h \circ g).$$
$h_{*}$ depends on the choice of $x_{0}$, so we denote $(h_{x_{0}})_{*}$ when necessary.

\subsection{Theorem.} If $i: (X, x_{0}) \to (X, x_{0})$ is the identity map, then $i_{*}$ is the identity homomorphism of $\pi_{1}(X, x_{0})$.

\subsection{Proof.} We compute that 
\begin{align*}
    i_{*}([f]) &= [i \circ f] \\
               &= [f].
\end{align*}

\subsection{Theorem.} If $h: (X, x_{0}) \to (Y, y_{0})$ and $k: (Y, y_{0}) \to (Z, z_{0})$ are continuous maps, then 
$$(k \circ h)_{*} = k_{*} \circ h_{*}.$$

\subsection{Proof.} We compute that 
\begin{align*}
    (k \circ h)_{*}([f]) &= [(k \circ h) \circ f] \\
                         &= [k \circ (h \circ f)] \\
                         &= k_{*}([h \circ f]) \\
                         &= (k_{*} \circ h_{*})([f]).
\end{align*}

\subsection{Corollary.} If $h: (X, x_{0}) \to (Y, y_{0})$ is a homeomorphism, then $h_{*}$ is an isomorphism of $\pi_{1}(X, x_{0})$ and $\pi_{1}(Y, y_{0})$.

\subsection{Proof.} Let $k$ be the inverse of $h$. Then $(k \circ h)_{*} = i_{*}$ is the identity homomorphism of $\pi_{1}(X, x_{0})$, and $(h \circ k)_{*} = j_{*}$ is the identity homomorphism of $\pi_{1}(Y, y_{0})$. So $h_{*}$ is an isomorphism.

\subsection{Definition.} Let $p: E \to B$ be a continuous surjective map. An open set $U$ of $B$ is said to be evenly covered by $p$ if $p^{-1}(U)$ is a disjoint union of open sets $\{V_{\alpha}\}$ such that $p \mid V_{\alpha}$ is a homeomorphism onto $U$ for each $\alpha$. The collection $\{V_{\alpha}\}$ is a partition of $p^{-1}(U)$ into slices.

\subsection{Definition.} Let $p: E \to B$ be a continuous surjective map. If each point $b \in B$ has a neighborhood $U$ that is evenly covered by $p$, then $p$ is said to be a covering map and $E$ is said to be a covering space of $B$.

\subsection{Definition.} A map $p: E \to B$ is a local homeomorphism if for every point $e \in E$ has a neighborhood $U$ that is homeomorphic to a neighborhood of $p(e)$ in $B$. A covering map is a local homeomorphism, but the converse does not hold necessarily.

\subsection{Example.} The map $p: \R_{+} \to \S^{1}$ defined by $p(x) = (\cos 2\pi x, \sin 2\pi x)$ is surjective and a local homomorphism, but it is not a coverign map. For the point $b_{0} = (1, 0)$ has no neighborhood $U$ that is evenly covered by $p$. This example also shows that the restriction of a covering map need not be a covering map.

\subsection{Theorem.} The map $p: \R \to S^{1}$ defined by $p(x) = (\cos 2\pi x, \sin 2\pi x)$ is a covering map.

\subsection{Proof.} Let $U$ be the subset of $S^{1}$ consisting of points with positive first coordinate, i.e., the open right semicircle. The preimage $p^{-1}(U)$ is the union of the open intervals $V_{n} = (n-1/4, n+1/4)$ for $n \in \Z$. The restriction $p \mid V_{n}$ is injective because $\sin \mid V_{n}$ is strictly monotonic, and it is surjective by the intermediate value theorem. $p \mid V_{n}$ is a continuous bijective map between a compact space and a Hausdorff space, so it is a homeomorphism of $V_{n}$ and $U$, so $U$ is evenly covered by $p$. A similar argument can be made for the left, upper, and lower semicircles. Thesse semicircles form an open cover of $S^{1}$, and each one is evenly covered by $p$, so $p$ is a covering map.

\subsection{Theorem.} Let $p: E \to B$ be a covering map. If $B_{0}$ is a subspace of $B$, and if $E_{0} = p^{-1}(B_{0})$, then the map $p_{0}: E_{0} \to B_{0}$ is a covering map, where $p_{0}$ is the restriction of $p$ to $E_{0}$.

\subsection{Proof.} Let $b_{0} \in B_{0}$. Let $U$ be a neighborhood of $b_{0}$ in $B$ that is evenly covered by $p$. Let $\{V_{\alpha}\}$ be a partion of $p^{-1}(U)$ into slices. Then $U \cap B_{0}$ is a neighborhood of $b_{0}$ in $B_{0}$, and the sets $\{V_{\alpha} \cap E_{0}\}$ are a partition of $p_{0}^{-1}(U \cap B_{0})$ into slices.

\subsection{Theorem.} If $p: E \to B$ and $p'L E' \to B'$ are covering maps, then 
$p \times p': E \times E' \to B \times B'$ is a covering map.

\subsection{Proof.} Given $b \in B$ and $b' \in B'$, let $U$ and $U'$ be neighborhoods of $b$ and $b'$, respectively, that are evenly covered by $p$ and $p$', respectively. Let $\{V_{\alpha}\}$ and $\{V_{\beta}'\}$ be partitions of $p^{-1}(U)$ and $(p')^{-1}(U')$ into slices. Then the preimage of $U \times U'$ is the union of all the sets $V_{\alpha} \times V_{\beta}'$. These are disjoint open sets in $E \times E'$, and each is mapped homeomorphically onto $U \times U'$ by $p \times p'$. So $p \times p'$ is a covering map.

\subsection{Example.} Let $T = S^{1} \times S^{1}$ be the torus. The map $p \times p: R^{2} \to T$ is a covering map, where $p$ is the covering map $p: \R to S^{1}$ of Theorem 3.48. Intuitively, each unit square of $\R^{2}$ is mapped homeomorphically onto a unit disk of $T$.

\subsection{Example.} Let $p \times p: \R^{2} \to T$ be the covering map of the previous example. Let $b_{0} = p(0)$, and let $B_{0}$ denote the figure-eight 
$$B_{0} = (S^{1} \times b_{0}) \cup (b_{0} \times S^{1}).$$
The preimage of $B_{0}$ is 
$$E_{0} = (\R \times \Z) \cup (\Z \times \R),$$
and the map $P_{0}: E_{0} \to B_{0}$ obtained by restricting $p \times p$ is a covering map.

\subsection{Example.} Let $p: \R \to S^{1}$ be the covering map of Theorem 3.48. Let $p \times i: R \times R_{+} \to S^{1} \times R_{+}$ be a covering map, where $i$ is the identity map of $R_{+}$. Let $x \times t \mapsto tx$ be a homeomorphism of $S^{1} \times R_{+}$ with $R^{2} \setminus \{0\}$. The composition yields a covering map $\R \times R_{+} \to R^{2} \setminus \{0\}$.

\subsection{Definition.} Let $p: E \to B$ be a map. If $f: X \to B$ is continuous, a lifting of $f$ is a map $\tilde{f}: X \to E$ such that $p \circ \tilde{f} = f$.

\subsection{Example.} Let $p: \R \to S^{1}$ be the covering map of Theorem 3.48. The path $f: I \to S^{1}$ defined by $f(t) = (\cos 2\pi ft, \sin 2\pi ft)$ lifts to the path $\tilde{f}: I \to \R$ defined by $\tilde{f}(t) = ft$.

\subsection{Lemma.} Let $p: E \to B$ be a covering map. Let $p_{e_{0}} = b_{0}$. Any path $f: I \to B$ beginning at $b_{0}$ has a unique lifting to a path $\tilde{f}: I \to E$ beginning at $e_{0}$.

\subsection{Proof.} Cover $B$ by open sets $U$ each of which is evenly covered by $p$.  Find a subdivision of $I$, say $s_{0}, \ldots, s_{n}$ such that every $f([s_{i}, s_{i+1}])$ lies in a $U$. (Use the Lebesgue number lemma.)

Define $\tilde{f}(0) = e_{0}$. Suppose that $\tilde{f}(s)$ is defined for $0 \leq s \leq s_{i}$. Define $\tilde{f}$ on $[s_{i}, s_{i+1}]$ as follows. The set $f([s_{i}, s_{i+1}])$ is contained in a $U$ that is evenly covered by $p$. Let $\{V_{\alpha}\}$ be a partition of $p^{-1}(U)$ into slices. Now $\tilde{f}(s_{i})$ lies in one of these sets, say $V_{0}$. Define $\tilde{f}(s)$ for $s_{i} \leq s \leq s_{i+1}$ by 
$$\tilde{f}(s) = (p \mid V_{0})^{-1}(f(s)).$$
$p \mid V_{0}: V_{0} \to U$ is a homeomorphism, so $tilde{f}$ is continuous on $[s_{i}, s_{i+1}]$. Hence, $\tilde{f}$ may be defined on $I$ by induction. It is continuous by the pasting lemma, and $p \circ \tilde{f} = f$.

Suppose that $\tilde{f}'$ is another lifting of $f$ beginning at $e_{0}$. Then $\tilde{f}'(0) = e_{0} = \tilde{f}(0)$. Suppose that $\tilde{f}'(s) = \tilde{f}(s)$ for all $0 \leq s \leq s_{i}$. By construction, $\tilde{f}'(s_{i})$ lies in the same slice $V_{0}$ as $\tilde{f}(s_{i})$. The map $p \mid V_{0}$ is a homeomorphism, so $\tilde{f}'(s_{i}) = \tilde{f}(s_{i})$. By induction, $\tilde{f}'(s) = \tilde{f}(s)$ for all $s \in I$, so the lifting is unique.

\subsection{Lemma.} Let $p: E \to B$ be a covering map. Let $p_{e_{0}} = b_{0}$. Let $F: I \times I \to B$ be continuous, with $F(0, 0) = b_{0}$. There is a unique lifting of $F$ to a continuous map $\tilde{F}: I \times I \to E$ such that $\tilde{F}(0, 0) = e_{0}$. If $F$ is a path homotopy, then $\tilde{F}$ is a path homotopy.

\subsection{Proof.} Given $F$, define $\tilde{F}(0, 0) = e_{0}$. Use Lemma 3.59 to define $\tilde{F}$ for $\{0\} \times I$ and for $I \times \{0\}$. Choose subdivisions $s_{0} < \ldots, < s_{m}$ and $t_{0} < \ldots < t_{n}$ small enough such that each rectangle $I_{i} \times J_{j} = [s_{i-1}, s_{i}] \times [t_{j-1}, t_{j}]$ is contained in an open set $U$ of $B$ that is evenly covered by $p$. (Use the Lebesgue number lemma.)

Given $i_{0}$ and $j_{0}$, assume that $\tilde{F}$ is defined on $A$, the union of $\{0\} \times I$ and $I \times \{0\}$ and all the previous rectangles. Assume that $\tilde{F}$ is a continuous lifting of $F \mid A$. Choose an open set $U$ of $B$ that is evnely covered by $p$ and contains $F(I_{i_{0}} \times J_{j_{0}})$. Let $\{V_{\alpha}\}$ be a partition of $p^{-1}(U)$ into slices. Now $\tilde{F}$ is already defined on $C = A \cap (I_{i_{0}} \times J_{j_{0}})$, which is connected, so $\tilde{F}(C)$ is connected, so it must lie entirely in, say, $V_{0}$. Let $p_{0}: V_{0} \to U$ denote the restriction of $p$ to $V_{0}$. Since $\tilde{F}$ is a lifting of $F \mid A$, for $x \in C$, $p_{0}(\tilde{F}(x)) = p(\tilde{F}(x)) = F(x)$, so that $\tilde{F}(x) = p_{0}^{-1}(F(x))$. Define 
$$\tilde{F}(x) = p_{0}^{1}(F(x))$$
for $x \in I_{i_{0}} \times J_{j_{0}}$. The lifting $\tilde{F}$ may be defined on $I \times I$ by induction. It is continuous by the pasting lemma, and $p \circ \tilde{F} = F$. It is unique for the same reason as for Lemma 3.59.

Suppose that $F$ is a path homotopy. The map $F$ carries $\{0\} \times I$ to $b_{0}$. Now $\tilde{F}$ is a lifting of $F$, so it carries this set to $p^{-1}(b_{0})$. But $p^{-1}(b_{0})$ has the discrete topology as a subspace of $E$. Since $\{0\} \times I$ is connected, $\tilde{F}(\{0\} \times I)$ is connected, so it must be a single point. So $\tilde{F}$ is a path homotopy.

\subsection{Theorem.} Let $p: E \to B$ be a covering map. Let $p_{e_{0}} = b_{0}$. Let $f$ and $g$ be two paths in $B$ from $b_{0}$ to $b_{1}$ and let $\tilde{f}$ and $\tilde{g}$ denote their respective liftings to paths in $E$, beginiing at $e_{0}$. If $f$ and $g$ are path homotpic, then $\tilde{f}$ and $\tilde{g}$ end at the same point of $E$ and are path homotopic.

\subsection{Proof.} Let $F: I \times I \to B$ be the path homotoy between $f$ and $g$. Then $F(0, 0) = b_{0}$. Let $\tilde{F}: I \times I \to E$ be the lifting of $F$ to $E$ such that $\tilde{F}(0, 0) = e_{0}$. By Lemma 3.61, $\tilde{F}$ is a path homotopy, so $\tilde{F}(\{0\} \times I) = \{e_{0}\}$ and $\tilde{F}(\{1\} \times I) = \{e_{1}\}$.

The restriction $\tilde{F} \mid (I \times \{0\})$ is a path on $E$ from $e_{0}$ that is the lifting of $F \mid (I \times \{0\})$. By the uniqueness of path liftings, $\tilde{F}(s, 0) = \tilde{f}(s)$. Similarly, $\tilde{F}(s, 1) = \tilde{g}(s)$. So $\tilde{f}$ and $\tilde{g}$ both end at $e_{1}$, and $\tilde{F}$ is a path homotopy between them.

\subsection{Definition.} Let $p: E \to B$ be a covering map. Let $b_{0} \in B$. Choose $e_{0}$ such that $p(e_{0}) = b_{0}$. Given an element $[f] \in \pi_{1}(B, b_{0})$, let $\tilde{f}$ be the lifting of $f$ to a path in $E$ that begins at $e_{0}$. Let $\phi([f])$ denote the end point $\tilde{f}(1)$ of $\tilde{f}$. Then $\phi: \pi_{1}(B, b_{0}) \to p^{-1}(b_{0})$ is a well-defined map. We call $\phi$ the lifting correspondence derived from the covering map $p$. It depends on the choice of $e_{0}$.

\subsection{Theorem.} Let $p: E \to B$ be a covering map. Let $p(e_{0}) = b_{0}$. If $E$ is path connected, then the lifting correspondence $\phi: \pi_{1}(B, b_{0}) \to p^{-1}(b_{0})$ is surjective. If $E$ is simply connected, then $\phi$ is bijective.

\subsection{Proof.} If $E$ is path connected, then, given $e_{1} \in p^{-1}(b_{0})$, there is a path $\tilde{f}$ from $e_{0}$ to $e_{1}$. Then $f = p \circ \tilde{f}$ is a loop in $B$ at $b_{0}$, and $\phi([f]) = e_{1}$. So $\phi$ is surjective.

Suppose that $E$ is simply connected. Let $[f]$ and $[g]$be two elements of $\pi_{1}(B, b_{0})$ such that $\phi([f]) = \phi([g])$. Let $\tilde{f}$ and $\tilde{g}$ be the liftings of $f$ and $g$, respectively, to path sin $E$ that begin at $e_{0}$. Then $\tilde{f}(1) = \tilde{g}(1)$. Since $E$ is simply connected, there is a path homotpy $\tilde{F}$ in $E$ between $\tilde{f}$ and $\tilde{g}$. So $p \circ \tilde{F}$ is a path homopty in $B$ between $f$ and $g$. So $[f] = [g]$, and $\phi$ is injective.

\subsection{Theorem.} The fundamental group of $S^{1}$ is isomorphic to $(\Z, +)$.

\subsection{Proof.} Let $p: \R \to S^{1}$ be the covering map of Theorem 3.48. Let $e_{0} = 0$ and let $b_{0} = p(e_{0})$. Then $p^{-1}(b_{0}) = \Z$. Since $\R$ is simply connected, the lifting correspondence $\phi: \pi_{1}(S^{1}, b_{0}) \to p^{-1}(b_{0})$ is bijective. Indeed, $\phi$ is a homomorphism.

Given $[f]$ and $[g]$ in $\pi_{1}(S^{1}, b_{0})$, let $\tilde{f}$ and $\tilde{g}$ be the liftings of $f$ and $g$, respectively. Let $m = \tilde{f}(1)$ and $n = \tilde{g}(1)$. So $\phi([f]) = m$ and $\phi([g]) = n$. Because $p(m + x) = p(x)$ for all $x \in \R$, 
$$\tilde{g}'(s) = m + \tilde{g}(s)$$
is the lifting of $g$ to $\R$ that begins at $m$. Then the product $\tilde{f} * \tilde{g}'$ is well-defined, and it is the lifting of $f * g$ to $\R$ that begins at $0$. The end point is $\tilde{g}' = m + n$. So $\phi([f] * [g]) = m + n$. so $\phi$ is a homomorphism and hence an isomorphism.

\subsection{Definition.} Let $G$ be a group and let $x \in G$ have the inverse $x^{-1}$. If the set of all elements $x^{m}$ for $m \in \Z$ equals $G$, then $G$ is said to be a cyclic group with generator $x$. $|G|$ is said to be the order of $G$. $G$ has infinite order if and only if it is isomorphic to the additive group of integers. $G$ has order $k$ if and only if it is isomorphic to the group $\Z / k$ of integers modulo $k$.

\subsection{Remark.} If $x$ is a generator of an inifite cyclic group $G$, and if $y$ is an element of an arbitrary group $H$, then there exists a unique homomorphism $h: G \to H$ such that $h(x) = y$ defined by $h(x^{n}) = y^{n}$ for all $n \in \Z$.

\subsection{Theorem.} Let $p: E \to B$ be a covering map. Let $p(e_{0}) = b_{0}$.

(a) The homomorphism $p_{*}: \pi_{1}(E, e_{0}) \to \pi_{1}(B, b_{0})$ is a monomorphism

(b) Let $H = p_{*}(\pi_{1}(E, e_{0}))$. The lifting correspondence $\phi$ induces an injective map 
$$\Phi: \pi_{1}(B, b_{0}) / H \to p^{-1}(b_{0})$$
of the collection of right cosets of $H$ into $p^{-1}(b_{0})$, which is bijective if $E$ is path connected.

(c) If $f$ is a loop in $B$ based at $b_{0}$, then $[f] \in H$ if and only if $f$ lifts to a loop in $E$ based at $e_{0}$.

\subsection{Proof.}

(a) Suppose that $\tilde{h}$ is a loop in $E$ baesd at $e_{0}$ and $p_{*}([\tilde{h}])$ is the identity element. Let $F$ be a path homotopy between $p \circ \tilde{h}$ and the constant loop. If $\tilde{F}$ is a lifting of $F$ to $E$ such that $\tilde{F}(0, 0) = e_{0}$, then $\tilde{F}$ is a path homotopy between $\tilde{h}$ and the constant loop at $e_{0}$.

(b) Given loops $f$ and $g$ in $B$, let $\tilde{f}$ and $\tilde{g}$ be the liftings of them to $E$ that begin at $e_{0}$. Then $\phi([f]) = \tilde{f}(1)$ and $\phi([g]) = \tilde{g}(1)$. We show that $\phi([g]) = \phi([g])$ if and only if $[f] \in H * [g]$.

Suppose that $[g] \in H * [g]$. Then $[g] = [h * g]$, where $h = p \circ \tilde{h}$ for some loop $\tilde{h}$ in $E$ based at $e_{0}$. Now the product $\tilde{h} * \tilde{g}$ is defined, and it is a lifting of $h * g$. Because $[f] = [h * g]$, the liftings $\tilde{f}$ and $\tilde{h} * \tilde{g}$ must end at the same point of $E$. So $\phi([f]) = \phi([g])$.

Now suppose that $\phi([f]) = \phi([g])$. Then $\tilde{f}$ and $\tilde{g}$ end at the same point of $E$. The product of $\tilde{f}$ and the reverse of $\tilde{g}$ is a loop $\tilde{h}$ in $E$ based at $e_{0}$. By direct computation, $[\tilde{h} * \tilde{g}] = [\tilde{f}]$. If $\tilde{F}$ is a path homotopy in $E$ between the loops $\tilde{h} * \tilde{g}$ and $\tilde{f}$, then $p \circ \tilde{F}$ is a path homotopy in $B$ between $h * g$ and $f$, where $h = p \circ \tilde{h}$. Thus, $[f] \in H * [g]$ as desired.

If $E$ is path connected, then $\phi$ is surjective, so $\Phi$ is surjective as well, and hence bijective.

(c) Injectivity of $\Phi$ implies that $\phi([f]) = \phi([g])$ if and only if $[f] \in H * [g]$. Now $\phi[f] = e_{0}$ if and only if $[f] \in H * [e_{0}] = H$. But $\phi([f]) = e_{0}$ if and only if $f$ lifts to a loop in $E$ based at $e_{0}$.

\subsection{Definition.} If $A \subseteq X$, a retraction of $X$ onto $A$ is a continuous map $r: X \to A$ such that $r \mid A$ is the identity map of $A$. If such an $r$ exists, then $A$ is said to be a retract of $X$.

\subsection{Lemma.} If $A$ is a retract of $X$, then the homomorphism of fundamental groups induced by the inclusion map $j: A \to X$ is injective.

\subsection{Proof.} If $r: X \to A$ is a retraction, then the composite map $r \circ j$ equals the identity map of $A$. It follows that $(r circ j)_{*} = r_{*} \circ j_{*}$ is the identity map of $\pi_{1}(A, a)$ by Theorem 3.38 and Theorem 3.40, so that $j_{*}$ is injective.

\subsection{Theorem.} (No-Retraction Theorem) There is no retraction of $B^{2}$ onto $S^{1}$.

\subsection{Proof.} If $S^{1}$ were a retract of $B^{2}$, then the homomorphism induced by $j: S^{1} \to B^{2}$ would be injective by Lemma 3.75. But the fundamental group of $S^{1}$ is nontrivial while the fundamental group of $B^{2}$ is trivial, a contradiction.

\subsection{Lemma.} Let $h: S^{1} \to X$ be a continuous map. Then teh following conditions are equivalent.

(a) $h$ is nulhomotopic.

(b) $h$ extends to a continuous map $k: B^{2} \to X$.

(3) $h_{*}$ is the trivial homomorphism of fundamental groups.

\subsection{Proof.} TODO!

\subsection{Corollary.} The inclusion map $j: S^{1} \to R^{2} \setminus \{0\}$ is not nulhomotopic. The identity map $i: S^{1} \to S^{1}$ is not nulhomotopic.

\subsection{Proof.} There is a retraction of $\R^{2} \setminus \{0\}$ onto $S^{1}$. Therefore, $j_{*}$ is injective and hence nontrivial. Similarly, $i_{*}$ is the indentity homomorphism and is hence nontrivial.

\subsection{Theorem.} Given a nonvanishing vector field on $B^{2}$, there exists a point of $S^{1}$ where the vector fiend points directly inward and a point of $S^{1}$ where the vector field points directly outward.

\subsection{Proof.} TODO!

\subsection{Theorem.} (Brouwer Fixed Point Theorem) If $f: B^{2} \to B^{2}$ is continuous, then there exists a point $x \in B^{2}$ such taht $f(x) = x$.

\subsection{Proof.} Suppose, by way of contradiction, that $f(x) \neq x$ for all $x \in B^{2}$. Then define $v(x) = f(x) - x$, which yields a nonvanishing vector field $(x, v(x))$ on $B^{2}$. But the vector field cannot point directly outward at any point $x$ of $X^{1}$, for otherwise $f(x) - x = ax$ for some $a > 0$, so $f(x) = (1 + a)x$ would lie outside the unit ball $B^{2}$, hence a contradiction.

\subsection{Theorem.} (Seifert-van Kampen Theorem) Suppose that $X = U \cup V$ where $U$ and $V$ are open sets of $X$. Suppose that $U \cap V$is path connected and that $x_{0} \in U \cap V$. Let $i$ and $j$ be the inclusion mappings of $U$ and $V$, respecitvely, into $X$. Then the images of the induced homomorphisms 
$$i_{*}: \pi_{1}(U, x_{0}) \to \pi_{1}(X, x_{0})$$
and 
$$j_{*}: \pi_{1}(V, x_{0}) \to \pi_{1}(X, x_{0})$$
generate $\pi_{1}(X, x_{0})$.

\subsection{Proof.} TODO!

\subsection{Remark.} The Seifert-van Kampen theorem says that any loop $f$ in $X$ based at $x_{0}$ is homotopic to a product of the form $g_{1} * (g_{2} * (\ldots * g_{n}))$ where each $g_{i}$ is a loop in $X$ based at $x_{0}$ that lies entirely in either $U$ or $V$.

\subsection{Corollary.} Suppose that $X = U \cup V$, where $U$ and $V$ are open sets of $X$. Suppose that $U \cap V$ is nonempty and path connected. If $U$ and $V$ are simply connected, then $X$ is simply connected.

\subsection{Proof.} The proof is obvious.

\subsection{Theorem.} If $n = 2, 3, \ldots$, then $S^{n}$ is simply connected.

\subsection{Proof.} TODO!

\end{document}
